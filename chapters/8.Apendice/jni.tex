\chapter{JNI}
\label{apend:jni}

JNI (\textit{Java Native Interface}) é uma \textit{framework} que permite que
aplicações em Java possam chamar e serem chamadas por aplicações nativas e
bibliotecas escritas em outras linguagem tal como C, C++ e Assembly.
O propósito dessa abordagem é oferecer a possibilidade de que programadores
possam implementar funcionalidades não disponibilizadas pela API padrão do Java
ou até mesmo melhorar as já implementadas seja por questão de desempenho,
segurança ou outros.

Para implementar os metodos nativos basta criar uma função com a estrutura
\ref{list:estruturaJNI}. Quando a JVM executar o metodo ela invocará a função
definida e passará os parametros conforme o esperado \cite{jniDoc}. 

	\begin{lstlisting}[caption=HelloWorld.java., label=list:estruturaJNI]	
		JNIEXPORT void JNICALL Java_ClassName_MethodName(JNIEnv *env, jobject obj) {
			/*Implement Native Method Here*/
		}
	\end{lstlisting}
	
\section{HelloWorld}

	Para criar o primeiro programa utilizando a JNI basta criar os arquivos
	\ref{list:helloWorldJava}, \ref{list:helloWorldH},
	\ref{list:helloWorldC} e \ref{list:make} e no console executar \cite{jniDoc}:
	
	\textit{chmod +x make.sh}
	
	\textit{./make.sh}


	\begin{lstlisting}[caption=HelloWorld.java., label=list:helloWorldJava]
	class HelloWorld {
			static {
				System.loadLibrary("HelloWorld");
	    }
	    
	    private native void print();
	        
	    public static void main(String[] args) {
	    	new HelloWorld().print();
	    }   
	}
	\end{lstlisting}
	
	\begin{lstlisting}[caption=HelloWorld.h., label=list:helloWorldH]
	
		/* DO NOT EDIT THIS FILE - it is machine generated */
		#include <jni.h>
		/* Header for class HelloWorld */
		 
		#ifndef _Included_HelloWorld
		#define _Included_HelloWorld
		#ifdef __cplusplus
		extern "C" {
		#endif
		/*
		 * Class:     HelloWorld
		 * Method:    print
		 * Signature: ()V
		 */
		JNIEXPORT void JNICALL Java_HelloWorld_print(JNIEnv *, jobject);
		 
		#ifdef __cplusplus
		}
		#endif
		#endif
	\end{lstlisting}
		
	\begin{lstlisting}[caption=HelloWorld.c., label=list:helloWorldC]
		#include "jni.h"
		#include <stdio.h>
		#include "HelloWorld.h"
		 
		JNIEXPORT void JNICALL Java_HelloWorld_print(JNIEnv *env, jobject obj) {
		    printf("Hello World!\n");
		    return;
		}
	\end{lstlisting}
	
	\begin{lstlisting}[caption=make.sh., label=list:make]
		#!/bin/sh
		export LD_LIBRARY_PATH=$LD_LIBRARY_PATH:.
		javah HelloWorld
		gcc -shared -Wl,-soname,HelloWorld -o libHelloWorld.so HelloWorld.c \
			-I/usr/lib/jvm/java-6-openjdk/include \
			-I/usr/lib/jvm/java-6-openjdk/include/linux
		javac HelloWorld.java
		java HelloWorld
	\end{lstlisting}
