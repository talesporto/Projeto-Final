\chapter{Processamento da imagem}
\label{apend:processamento}

As etapas de processamento da imagem utilizadas nesse trabalho consistem na conversão em escala de cinza, corte, redimensionamento e equalização da mesma. Neste apêndice são mostrado trechos de códigos que implementam tais etapas utilizando a biblioteca \textit{OpenCV}.

\section{Escala de Cinza}

		\begin{lstlisting}[caption=Conversão de uma imagem para escala de cinza., label=list:grey-scale]
		IplImage* imageToGreyscale(const IplImage *imageSrc) {
			IplImage *imageGrey;

			// cria uma nova imagem do mesmo tamanho que a imagem passada como parametro e de 1 canal.
			imageGrey = cvCreateImage(cvGetSize(imageSrc), IPL_DEPTH_8U, 1);

			//converte o espaco de cor de imageSrc para o espaco de cor (cinza) da imageGrey
			cvCvtColor(imageSrc, imageGrey, CV_BGR2GRAY);

			return imageGrey;
		}
		\end{lstlisting}

\section{Corte da Imagem}

	\begin{lstlisting}[caption=Corte de uma imagem., label=list:crop]
		IplImage* crop(const IplImage *img, const CvRect region) {
			IplImage *imageTmp;
			IplImage *imageRGB;
			CvSize size;
			size.height = img->height;
			size.width = img->width;

			if (img->depth != IPL_DEPTH_8U) {
				printf("ERROR: img->depth de %d desconhecido ao inves de 8 bits por pixel passado a crop().\n", img->depth);
				exit(1);
			}

			// cria uma nova imagem (colorida ou em cinza) e copia os conteudos da imagem nela.
			imageTmp = cvCreateImage(size, IPL_DEPTH_8U, img->nChannels);
			cvCopy(img, imageTmp, NULL);

			// cria uma nova imagem da regiao detectada
			// seta a região de interesse que é ao redor da face
			cvSetImageROI(imageTmp, region);

			//copia a imagem de interesse na nova iplImage (imageRGB) e a retorna
			size.width = region.width;
			size.height = region.height;
			imageRGB = cvCreateImage(size, IPL_DEPTH_8U, img->nChannels);
			cvCopy(imageTmp, imageRGB, NULL);//copia somente a regiao

			cvReleaseImage(&imageTmp);
			return imageRGB;
		}
	\end{lstlisting}

\section{Redimensionamento}

	\begin{lstlisting}[caption=Redimensionamento de uma imagem., label=list:resize]
		IplImage* resize(const IplImage *origImg, int newWidth, int newHeight) {
			IplImage *outImg = 0;
			int origWidth, origHeight;

			if (origImg) {
				origWidth = origImg->width;
				origHeight = origImg->height;
			}

			if (newWidth <= 0 || newHeight <= 0 || origImg == 0 || origWidth <= 0 || origHeight <= 0) {
				printf("ERROR: Tamanho %dx%d desejado para imagem invalido\n.", newWidth, newHeight);
				exit(1);
			}

			// modifica as dimensoes da imagem, mesmo se a proporcao mude
			outImg = cvCreateImage(cvSize(newWidth, newHeight), origImg->depth, origImg->nChannels);
			cvResetImageROI((IplImage*) origImg);
			if (newWidth > origImg->width && newHeight > origImg->height) { // aumentar a imagem
				//CV_INTER_LINEAR muito boa para aumentar a imagem
				cvResize(origImg, outImg, CV_INTER_LINEAR); 

			} else { //diminuir a imagem
				//CV_INTER_AREA muito boa para diminuir a imagem, porem pessima para aumenta-la
				cvResize(origImg, outImg, CV_INTER_AREA);
			}

			return outImg;
		}
	\end{lstlisting}

\section{Equalização}

	\begin{lstlisting}[caption=Equalização de uma imagem., label=list:equalizacao]
		//cria uma imagem limpa na escala de cinza
		equalizedImg = cvCreateImage(cvGetSize(sizedImg), 8, 1); 

		// metodo que realiza "equalização de histograma"
		// normalizando o brilho e aumentando o contraste
		cvEqualizeHist(sizedImg, equalizedImg);
	\end{lstlisting}	

	% TODO : remover os acentos dentro dos lstlisting