\textual

\chapter{Introdução}
	
A computação ubíqua a tempos vem sendo tema de diversas pesquisas ao redor do mundo. Mark Weiser diz que o computador do futuro deve ser algo invisível \cite{weiser1} \cite{weiser2}, o que proporciona ao usuário melhor foco na tarefa e não na ferramenta. A computação ubíqua tenta proporcionar a "invisibilidade" desta ferramenta. Como aconteceu com o motor, o computador também vive um momento "down-size", diminuindo cada vez mais o seu tamanho e se acoplando aos objetos do dia-a-dia.

O SmartSpace é um ambiente onde a computação ubíqua acontece em sua totalidade. Esse ambiente provê ao usuário uma melhor forma de interagir com os computadores do ambiente usando diversas tecnologias que estimulam a interatividade natural entre usuário e computador esse ambiente é o que [Gregory Abowd] prevê para o futuro.

Para conseguir uma boa interação entre as diversas peças que compõem o SmartSpace é necessário que se tenha a disposição informações de contexto,  como quem está no ambiente, onde está, o que está fazendo e outras que ajudam o sistema a definir o melhor ajuste dos equipamentos. Com uma base rica de informações de contexto contendo os perfis dos usuários garantimos uma maior acurácia na tomada de decisões. 

Informações de contexto como essas são complicadas de se obter devido a alta dinamicidade do ambiente, no qual usuários entram e saem a todo momento e interagem com diversos equipamentos.

Kinect.

Para resolver esse impasse surgiu a idéia de usar o Kinect como ferrmenta de localização e identificação.
