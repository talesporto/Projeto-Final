\chapter{Conclusões}
\label{cap:conclusao}

O trabalho teve como objetivo a implementação de um sistema que utiliza o sensor \textit{Kinect} e tem como requisitos alvo o rastreamento dos usuário no ambiente, a localização e identificação dos usuários rastreados e prover tais informações ao Middleware \textit{uOS}.

O rastreamento é fundamental para o funcionamento correto do Sistema TRUE. Ele é responsável por detectar e rastrear os usuários servindo como base para que novas informações possam ser obtidas. O Sistema TRUE se mostrou eficaz na detecção de novos usuários que são detectados antes mesmo de entrarem por completo no campo de visão e se revelou robusto em casos de oclusões parciais. Contudo, casos de oclusões totais prejudicam a eficácia do sistema. Além disso, o campo de visão se mostrou limitado rastreando no máximo 5 pessoas simultâneamente à uma distância máxima de 4,057 metros.

O Sistema TRUE estimava a posição dos usuários por meio de coordenadas em relação ao sensor \textit{Kinect}. Portanto, fixando a posição do sensor é possível obter a localização física dos usuários no ambiente. As coordenadas obtidas se mostraram bem precisas e confiáveis apresentando erros de poucos centímetros que aumentam juntamente com a distância entre o usuário e o sensor. 



% rastrea
% local
% inde

\section{Trabalhos Futuros}

Era esperado ao término deste trabalho que o sistema rastreasse, identificasse e localizasse os usuários no ambiente inteligente. No entanto, foi mostrado que há limitações referentes ao rastreamento e identificação. Dada a implementação do Sistema TRUE e suas funcionaldiades da forma que foi definida, os próximos passos seriam:

\begin{itemize}
	\item ampliação do número de sensores \textit{Kinect} utilizados pelo sistema para que se torne mais robusto aos problemas causados pela oclusão e pelo campo de visão limitado.
	\item utilização de técnicas que torne o reconhecimento facial mais robusto em relação as constantes variações de poses e ângulos das faces capturadas.
\end{itemize}




