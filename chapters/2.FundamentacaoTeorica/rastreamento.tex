\section {Rastreamento}	

	Rastreamento de objetos é uma importante tarefa do campo da computação visual. A ploriferação de computadores com um alto poder computacional, a disponibilidade de camêras de alta qualidade e preço acessível e a crescente necessidade de sistemas automáticos de análise de vídeos têm gerado um grante interesse em algoritmos de rastreamento de objetos~\cite{yilmaz}.

	Basicamente, rastreamento pode ser definido como o problema de estimar a trajetória de um objeto em um plano de imagem a medida em que se move na cena. Em outras palavras, um rastreador atribui \textit{labels} para os objetos monitorados em diferentes quadros de um vídeo~\cite{yilmaz}.

	A detecção e o rastreamento de pessoas tem um grande potencial em aplicações em domínios tão diversos como animação, interação humano-computador, vigilância automatizada (monitorar uma cena para detectar atividades suspeitas), entre outros. Por esta razão, tem havido um crescimento notável na investigação deste problema.

	O rastreamento de pessoas em um ambiente é considerada como uma tarefa complexa devido a:

		\begin{enumerate}
			\item complexidade do corpo humano;
			\item alta dinamicidade do ambiente;
			\item ruído nas imagens~\cite{yilmaz};
			\item complexidade do movimento das pessoas;
			\item oclusões parciais ou totais de pessoas;
			\item variação na iluminação do ambiente~\cite{yilmaz};
			\item processamento em tempo-real~\cite{yilmaz};
		\end{enumerate}

	Várias abordagens para rastreamento de objetos já foram propostas. Basicamente, elas se diferem na forma que tratam as seguintes perguntas~\cite{yilmaz}: 
		
		\begin{itemize}
			\item Qual representação do objeto é adequada para o rastreamento?
			\item Quais características na imagem devem ser utilizadas?
			\item Como o movimento, aparência e a forma do objeto deve ser modelada? 
		\end{itemize}

	As respostas para estas perguntas dependem do contexto/ambiente onde o rastreamento será utilizado e do uso final para o qual as informações de rastreamento~\cite{yilmaz}.

	Basicamente, o processo de rastreamento pode ser dividido em duas etapas:

		\begin{enumerate}
			\item Detecção do objeto;
			\item Rastreamento do objeto detectado;
		\end{enumerate}

	Antes de falarmos mais sobre cada uma dessas etapas e os métodos existentes para cada, vamos falar sobre as maneiras existentes de representar os objetos rastreados e sobre as características nas imagens que podem ser utilizadas.

%%%%%%%%%%%%%%%%%%%%%%%%%%%%%%%%%%%%%%%%%%%%%%%%%%%%%%%%%%%%%%%%%%%%%%%%%%%%%%%%%%%%%%%%%%%%%%%%%%%%%%%%%%%%%%%%%%%%%%%%%%%%%%%%%%%%%%%%%%%%%%%%%%%%%%%%%%%%%%%%%%%%%%%%%%%%%%%%%%%%%%%%%%%%%%%%%%%%%%%%%%%%%%%%%%%%%%%%%%%%%%%%%%%%%%%%%%%%%%%%%%%%%%%%%%%%%%%%%%%%%%%%%%%%%%%%%%%%%%%%%%%%%%%%%%%%%%%%%%%%%%%%%%%%%%%%%%%%%%%%%%%%%%%%%%%%%%%%%%%%%%%%%%%%%%%%%%%%%%%%%%%%%%%%%%%%%%%%%%%%%%%%%%%%%%%%%%%%%%%%%%%%%%%% 

\subsection{Representação do Objeto}

	Nos sistemas de rastreamento, os objetos rastreados devem ser representados de alguma maneira. Geralmente, as representações são baseados em suas formas. Existe uma forte relação entre a representação do objeto e o algoritmo de rastreamento escolhido. A representação é escolhida baseada no domínio da aplicação e as mais utilizadas são~\cite{yilmaz}:

	\begin{figure}[hbt]
		\begin{center}
			\includegraphics[scale=0.5]{figuras/2.FundamentacaoTeorica/representacao.png}
		\end{center}
		\caption{Representações de objetos rastreados. (a) Centróide, (b) múltiplos pontos, (c) representação retangular, (d) representação elíptica, (e) representação de múltiplas partes, (f) esqueleto do objeto, , (g) conotorno completo do objeto, (h) contorno do objeto por pontos, (i) silhueta do objeto~\cite{yilmaz}}
		\label{representacao}
	\end{figure}

	\begin{itemize}
		\item \textbf{Pontos:} o objeto é representado por um ponto, como por exemplo a centróide da Figura \ref{representacao}(a), ou por vários pontos, como por exemplo na Figura \ref{representacao}(b). Essa representaçao é mais adequada para rastreamento de objetos que ocupam uma pequena região na imagem;
		\item \textbf{Formas geométricas primitivas:} o objeto é representado por formas geométricas simples como um retângulo e uma elipse, como mostrados nas Figuras \ref{representacao}(c) e (d). Essa representação é mais adequada para simples objetos rígidos;
		\item \textbf{Silhueta e Contorno:} representação por contorno define os limites de um objeto, como mostrado nas Figura \ref{representacao}(g) e (h). A região interna do contorno é chamada de Silhueta, como mostrado na Figura \ref{representacao}(i). Essa represetação é mais adequada para rastrear objetos complexos de forma não rígida;
		\item \textbf{Modelos de formas articuladas:} objetos articulados são compostos por partes do corpo que se ligam por meio de juntas. Para representar objetos articulados, utiliza-se figuras geométricas para cada parte do corpo, como mostrado na Figura \ref{representacao}(e);
		\item \textbf{Modelos de Esqueletos:} modelos de esqueletos são extraídos do objeto rastreado, como mostrado na Figura \ref{representacao}(f). Essa representação pode ser utilizada tanto para objetos articulados rígidos quanto não rígidos;
	\end{itemize}

	Para rastreamento de pessoas a representação por meio de contorno ou silhuetas são as mais adequadas~\cite{yilmaz}.

%%%%%%%%%%%%%%%%%%%%%%%%%%%%%%%%%%%%%%%%%%%%%%%%%%%%%%%%%%%%%%%%%%%%%%%%%%%%%%%%%%%%%%%%%%%%%%%%%%%%%%%%%%%%%%%%%%%%%%%%%%%%%%%%%%%%%%%%%%%%%%%%%%%%%%%%%%%%%%%%%%%%%%%%%%%%%%%%%%%%%%%%%%%%%%%%%%%%%%%%%%%%%%%%%%%%%%%%%%%%%%%%%%%%%%%%%%%%%%%%%%%%%%%%%%%%%%%%%%%%%%%%%%%%%%%%%%%%%%%%%%%%%%%%%%%%%%%%%%%%%%%%%%%%%%%%%%%%%%%%%%%%%%%%%%%%%%%%%%%%%%%%%%%%%%%%%%%%%%%%%%%%%%%%%%%%%%%%%%%%%%%%%%%%%%%%%%%%%%%%%%%%%%%%

\subsection{Características para rastreamento}

	A seleção das características é uma tarefa crítica para o rastreamento e está fortemente relacionada com a representação do objeto. Em geral, a seleção procura as características mais singulares para que o objeto rastreado seja facilmente distinguido. As características mais comuns utilizadas autlamente são~\cite{yilmaz}:

	\begin{itemize}
		\item \textbf{Cor:} a cor do objeto é influênciada principalmente por duas características: a distribuição da iluminação e a propriedade de reflectância do objeto. Geralmente, o \textit{RGB} geralmente é utilizado para representar a cor;

		\item \textbf{Borda:} Os limites de um objeto gera uma grande variação na intensidade da imagem. A detecção por meio das bordas é utilizadado para identificar essas variações. As bordas são menos sensíveis a variações na iluminação comparado com as cores. Os algoritmos que detectam as bordas do objeto geralmente as utiliza para representação do mesmo;

		\item \textbf{};

		\item \textbf{};

	\end{itemize}





\subsection{Detecção de objeto}

	Todo metodo de Rastreamento requer um mecanismo de detecção de objetos seja comparando cada frame com seu posterior ou quando o objeto aparece pela primeira vez no video.




	\begin{itemize}
		\item \textbf{Detector de pontos:} esses detectores são usados para encontrar pontos de interesses dentro da imagem que tem uma expressiva textura na sua respectiva localização. Pontos de interesse são amplamente usados no contexto do movimento, estereo e o problema do rastreamento. A qualidade desejável para o ponto de interesse é que seja inveriante diante das mudanças de iluminação e ângulo da camera.















% Adaptive Particle Filter with Body Part Segmentation for Full Body Tracking
% 	The existing full body tracking algorithms can be classified into two types. One is the method with monocular or multi-view images [2, 3, 4, 5], and the other is the method with 3D reconstructed data [1, 6, 7, 8]. The approaches with 2D images have the advantage that they work with a simple hardware setup. However, self-occlusion makes the 2D tracking problem hard for arbitrary movements. Thus the existing systems assume some a-priori knowledge of the type of movement and/or the viewpoint under which it is observed [9]. With the 3D reconstructed data we can process more kinds of action and


% Tracking Human Motion in Structured Environments Using a Distributed-Camera System
% 	2	SINGLE VIEW TRACKING
% 	Tracking from a single view includes two major components: preprocessing and feature correspondence between consecutive frames. Three stages of preprocessing are performed:
% 	Segmenting the moving objects from the still background, 
% 	Distinguishing human subjects from other segmented nonbackground objects, and Extracting features from the segmented human subjects.