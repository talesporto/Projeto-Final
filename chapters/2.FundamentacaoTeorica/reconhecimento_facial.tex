\textual

\chapter{Localização e Reconhecimento Facial}

	Algo explicando o que terá nesse capítulo.


\section{Localização}

\section{Reconhecimento Facial}

Biometria é uma tecnologia utilizada para identificar uma pessoa baseada em características pessoais. Teoricamente, qualquer característica física/comportamental pode ser utilizada para identificação caso siga alguns dos seguintes requisitos \cite{milene}: 

	\begin{enumerate}
		\item \textbf{universidade}: qualquer pessoa pode ser avaliada sobre essa característica;
		\item \textbf{singularidade}: dada duas pessoas distinas, elas não podem ter a mesma característica;
		\item \textbf{permanência}: a característica não pode mudar de acordo com o tempo;
		\item \textbf{exigibilidade}: significa que a característica pode ser mensurada quantitativamente;
	\end{enumerate}

Porém, na prática também são consiredaos outros requisitos \cite{milene}:

	\begin{enumerate}
		\item \textbf{desempenho}: o processo de identificação deve apresentar um resultado aceitavel;
		\item \textbf{aceitação}: indica em que ponto as pessoas estão dispostas a ceitar o sistema biométrico;
		\item \textbf{evasão}: refere a facilidade de ser adulterado;
	\end{enumerate}

O reconhecimento facial é uma das nove características biométricas utilizadas para reconhecimento individual \cite{milene}.
	
	\begin{center}
	\begin{tabular}{|c|c|c|c|c|c|c|c|}
		\hline Biometria & Universidade & Singularidade & Permanência & Exigibilidade & Desempenho & Aceitação & Evasão \\
		\hline Face & Alta & Baixa & Médio & Alta & Baixa & Alta & Baixa\\
		\hline Digital & Médio & Alta & Alta & Médio & Alta & Médio &  Alta\\
		\hline Geometria da Mão & Médio & Médio & Médio & Alta & Médio & Médio & Médio\\
		\hline "Hand Vein" & Médio & Médio & Médio & Médio & Médio & Médio & High\\
		\hline Iris & Alta & Alta & Alta & Médio & Médio & Médio & Alta\\
		\hline "Retina Scan" & Alta & Alta & Médio & Baixa & Alta & Baixa & Alta\\
		\hline Assinatura & Baixa & Baixa & Baixa & Alta & Baixa & Alta & Baixa \\
		\hline Voz & Médio & Baixa & Baixa & Médio & Baixa & Alta & Baixa \\
		\hline Termograma & Alta & Alta & Baixa & Alta & Médio & Alta & Alta \\
		\hline
	\end{tabular}
	\end{center}











































