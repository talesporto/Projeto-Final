
\chapter{Localização e Reconhecimento Facial}

	Algo explicando o que terá nesse capítulo.


\section{Localização}

\section{Reconhecimento Facial}

Biometria é uma tecnologia utilizada para identificar uma pessoa baseada em características pessoais. Teoricamente, qualquer característica física/comportamental pode ser utilizada para identificação caso siga alguns dos seguintes requisitos \cite{milene}: 

	\begin{enumerate}
		\item \textbf{universidade}: qualquer pessoa pode ser avaliada sobre essa característica;
		\item \textbf{singularidade}: dada duas pessoas distinas, elas não podem ter a mesma característica;
		\item \textbf{permanência}: a característica não pode mudar de acordo com o tempo;
		\item \textbf{exigibilidade}: significa que a característica pode ser mensurada quantitativamente;
	\end{enumerate}

Porém, na prática também são considerados outros requisitos \cite{milene}:

	\begin{enumerate}
		\item \textbf{desempenho}: o processo de identificação deve apresentar um resultado aceitável;
		\item \textbf{aceitação}: indica em que ponto as pessoas estão dispostas a aceitar o sistema biométrico;
		\item \textbf{evasão}: refere a facilidade de ser adulterado;
	\end{enumerate}

Novas técnicas de reconhecimento por meio de faces, íris, retina e voz, entre outras, têm sido abordadas para aplicações em sistemas de reconhecimento automático \cite{bolle} \cite{saocarlos}. O reconhecimento facial é, apenas, uma das nove características biométricas utilizadas atualmente \cite{milene}.
	
	\begin{center}
	\begin{tabular}{|c|c|c|c|c|}
		\hline Biometria & Universidade & Singularidade & Permanência & Exigibilidade \\
		\hline Face & Alta & Baixa & Média & Alta \\
		\hline Digital & Média & Alta & Alta & Média \\
		\hline Geometria da Mão & Média & Média & Média & Alta \\
		\hline "Hand Vein" & Média & Média & Média & Média \\
		\hline Iris & Alta & Alta & Alta & Média \\
		\hline "Retina Scan" & Alta & Alta & Média & Baixa \\
		\hline Assinatura & Baixa & Baixa & Baixa & Alta\\
		\hline Voz & Média & Baixa & Baixa & Média \\
		\hline Termograma & Alta & Alta & Baixa & Alta \\
		\hline
	\end{tabular}


	\begin{tabular}{|c|c|c|c|}
		\hline Biometria & Desempenho & Aceitação & Evasão \\
		\hline Face & Baixa & Alta & Baixa\\
		\hline Digital & Alta & Média &  Alta\\
		\hline Geometria da Mão & Média & Média & Média\\
		\hline "Hand Vein" & Média & Média & Alta\\
		\hline Iris & Média & Média & Alta\\
		\hline "Retina Scan" & Alta & Baixa & Alta\\
		\hline Assinatura & Baixa & Alta & Baixa \\
		\hline Voz & Baixa & Alta & Baixa \\
		\hline Termograma & Média & Alta & Alta \\
		\hline
	\end{tabular}
	\end{center}


Um dos motivos que icentivou os diversos estudos sobre reconhecimento facial são as vantagens que o mesmo possui em relação a impressão digital e a íris.  No reconhecimento por impressão digital, a desvantagem consiste no fato que nem todas as pessoas possuem uma impressão digital com "qualidade" suficiente para ser reconhecida por um sistema. Já o reconhecimento por íris apresenta uma alta confiabilidade e larga variação, sendo estável pela vida toda. Porém, a desvantagem está relacionada ao modo de captura da íris que necessita de uma alinhamento entre a câmera e os olhos da pessoa \cite{saocarlos}.

No anos 70, os estudos do reconhecimento facial eram baseados sobre atributos faciais mensuráveis como olhos, nariz, sobrancelhas, bocas, entre outros. Porém, os recursos computacionais eram escassos e os algoritmos de extração de características eram ineficiêntes. Então, as pesquisas na área ressurgiram nos anos 90, inovando os métodos existentes \cite{Hon98}\cite{saocarlos}.

A maioria dos sitemas de reconhecimento facial são compostos por tarefas preliminares, como detecção e a segmentação \cite{Sun98}. 
Algumas destas utilizam diversas tarefas que compõe o processamento facial que, por sua vez, é composto por diferentes tarefas que variam entre as aplicações, como por exemplo \cite{saocarlos}:
	\begin{enumerate}
		\item \textbf{Classificação}: classificação de uma face visualmente com base em categorias como pele, olhos \cite{Abd95};
		\item \textbf{Identificação}: verifica se uma face pertence a um conjunto de faces conhecidas \cite{Cell99};
		\item \textbf{Reconhecimento}: diz se uma face é "familiar" ou não \cite{Abd95};
		\item \textbf{Detecção}: em uma imagem qualquer, se reconhece o local onde as faces estão \cite{Cell99};
		\item \textbf{Segmentação}: identifica as partes que compõe uma face \cite{Cell99};
		\item \textbf{Representação}: seleciona as informações de uma face que serão utilizadas para representá-las \cite{Cell99};
	\end{enumerate}






