
\chapter{Localização e Reconhecimento Facial}

	Algo explicando o que terá nesse capítulo.


\section{Localização}

\section{Reconhecimento Facial}

As abordagens de identificação pessoal que utilizam ``alguma coisa que você sabe'', como Número de Indetificação Pessoal (PIN - ``Personal Identification Number''), ou ``alguma coisa que você tenha'', como um cartão de identificação, não são confiáveis o suficiente para satisfazer os requisitos de segurança de um sistema de transações eletrônicas porque não têm a capacidade de diferenciar um usuário legítimo de um impostor que adiquiriu de forma ilegal o privilégio de acesso \cite{hong}.

Biometria é uma tecnologia utilizada para identificação de um indivíduo baseado em suas características físicas ou comportamental, baseia-se em ``alguma coisa que você é ou faz'' para realizar a identificação e, por isso, tem a capacidade de diferenciar entre um indivíduo legítimo de um impostor \cite{hong}. Teoricamente, qualquer característica física/comportamental pode ser utilizada para identificação caso siga alguns dos seguintes requisitos \cite{milene}: 

	\begin{enumerate}
		\item \textbf{universidade}: qualquer pessoa pode ser avaliada sobre essa característica;
		\item \textbf{singularidade}: dada duas pessoas distinas, elas não podem ter a mesma característica;
		\item \textbf{permanência}: a característica não pode mudar de acordo com o tempo;
		\item \textbf{exigibilidade}: significa que a característica pode ser mensurada quantitativamente;
	\end{enumerate}

Porém, na prática também são considerados outros requisitos \cite{milene}:

	\begin{enumerate}
		\item \textbf{desempenho}: o processo de identificação deve apresentar um resultado aceitável;
		\item \textbf{aceitação}: indica em que ponto as pessoas estão dispostas a aceitar o sistema biométrico;
		\item \textbf{evasão}: refere a facilidade de ser adulterado;
	\end{enumerate}

Novas técnicas de reconhecimento por meio de faces, íris, retina e voz, entre outras, têm sido abordadas para aplicações em sistemas de reconhecimento automático \cite{bolle,saocarlos}. O reconhecimento facial é, apenas, uma das nove características biométricas utilizadas atualmente \cite{milene}. Nas tabelas \ref{tabelaRequisitosTeoricos} e \ref{tabelaRequisitosPraticos} são mostradadas as noves características biométricas mais utilizadas e seus respectivos comportamentos baseados nos requisitos mencionados acima.
		
	\begin{table}[htbp]
		\begin{center}
			\caption{Requisitos teóricos para algoritmos de reconhecimento facial \cite{milene}.}
			\begin{tabular}{|c|c|c|c|c|}
				\hline \bf Biometria & \bf Universidade & \bf Singularidade & \bf Permanência & \bf Exigibilidade \\
				\hline \hline \bf Face & Alta & Baixa & Média & Alta \\
				\hline \bf  Digital & Média & Alta & Alta & Média \\
				\hline \bf Geometria da Mão & Média & Média & Média & Alta \\
				\hline \bf ``Hand Vein'' & Média & Média & Média & Média \\
				\hline \bf Iris & Alta & Alta & Alta & Média \\
				\hline \bf ``Retina Scan'' & Alta & Alta & Média & Baixa \\
				\hline \bf Assinatura & Baixa & Baixa & Baixa & Alta\\
				\hline \bf Voz & Média & Baixa & Baixa & Média \\
				\hline \bf Termograma & Alta & Alta & Baixa & Alta \\
				\hline
			\end{tabular}
		\end{center}
		\label{tabelaRequisitosTeoricos}
	\end{table}

	\begin{table}[htbp]
		\begin{center}
			\caption{Requisitos práticos para algoritmos de reconhecimento facial \cite{milene}.}
			\begin{tabular}{|c|c|c|c|}
				\hline \bf Biometria & \bf Desempenho & \bf Aceitação & \bf Evasão \\
				\hline \hline \bf Face & Baixa & Alta & Baixa\\
				\hline \bf Digital & Alta & Média &  Alta\\
				\hline \bf Geometria da Mão & Média & Média & Média\\
				\hline \bf ``Hand Vein'' & Média & Média & Alta\\
				\hline \bf Iris  & Média & Média & Alta\\
				\hline \bf ``Retina Scan'' & Alta & Baixa & Alta\\
				\hline \bf Assinatura & Baixa & Alta & Baixa \\
				\hline \bf Voz & Baixa & Alta & Baixa \\
				\hline \bf Termograma & Média & Alta & Alta \\
				\hline
			\end{tabular}
		\end{center}
		\label{tabelaRequisitosPraticos}
	\end{table}

Um dos motivos que icentivou os diversos estudos sobre reconhecimento facial são as vantagens que o mesmo possui em relação a impressão digital e a íris.  No reconhecimento por impressão digital, a desvantagem consiste no fato que nem todas as pessoas possuem uma impressão digital com ``qualidade'' suficiente para ser reconhecida por um sistema. Já o reconhecimento por íris apresenta uma alta confiabilidade e larga variação, sendo estável pela vida toda. Porém, a desvantagem está relacionada ao modo de captura da íris que necessita de uma alinhamento entre a câmera e os olhos da pessoa \cite{saocarlos}.

No anos 70, os estudos do reconhecimento facial eram baseados sobre atributos faciais mensuráveis como olhos, nariz, sobrancelhas, bocas, entre outros. Porém, os recursos computacionais eram escassos e os algoritmos de extração de características eram ineficiêntes. Então, as pesquisas na área ressurgiram nos anos 90, inovando os métodos existentes \cite{hong, saocarlos}.








