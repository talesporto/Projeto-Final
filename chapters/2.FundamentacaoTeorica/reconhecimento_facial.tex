\section{Reconhecimento Facial}

Um dos motivos que icentivou os diversos estudos sobre reconhecimento facial são as vantagens que o mesmo possui em relação a impressão digital e a íris.  No reconhecimento por impressão digital, a desvantagem consiste no fato que nem todas as pessoas possuem uma impressão digital com ``qualidade'' suficiente para ser reconhecida por um sistema. Já o reconhecimento por íris apresenta uma alta confiabilidade e larga variação, sendo estável pela vida toda. Porém, a desvantagem está relacionada ao modo de captura da íris que necessita de uma alinhamento entre a câmera e os olhos da pessoa \cite{saocarlos}.

No anos 70, os estudos do reconhecimento facial eram baseados sobre atributos faciais mensuráveis como olhos, nariz, sobrancelhas, bocas, entre outros. Porém, os recursos computacionais eram escassos e os algoritmos de extração de características eram ineficiêntes. Então, as pesquisas na área ressurgiram nos anos 90, inovando os métodos existentes \cite{hong, saocarlos}.