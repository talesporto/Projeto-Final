\section{Reconhecimento Facial}

O reconhecimento facial é uma das atividades mais comuns realizadas diariamente por seres vivos dotados de certa inteligência. Esta atividade corriqueira vem despertando o interesse de pesquisadores que trabalham com Visão Computacional e Inteligência Artificial. O objetivo desses pesquisadores é de construir sistemas artificiais capazes de realizar o reconhecimento de faces humanas e a partir desta capacidade construir os mais diferentes tipos de aplicações: sistemas de vigilância, controles de acesso, definções automáticas de perfis, entre outros \cite{oliveira}.

Um dos motivos que icentivou os diversos estudos sobre reconhecimento facial são as vantagens que o mesmo possui em relação a impressão digital e a íris.  No reconhecimento por impressão digital, a desvantagem consiste no fato que nem todas as pessoas possuem uma impressão digital com ``qualidade'' suficiente para ser reconhecida por um sistema. Já o reconhecimento por íris apresenta uma alta confiabilidade e larga variação, sendo estável pela vida toda. Porém, a desvantagem está relacionada ao modo de captura da íris que necessita de uma alinhamento entre a câmera e os olhos da pessoa \cite{saocarlos}. 

Basicamnente existem duas particularidades que fazem da face uma característica biométrica bastante atrativa \cite{drovetto}:

	\begin{enumerate}
		\item A aquisição da face é feita de forma fácil e não-intrusiva;
		\item Possui uma baixa privacidade de informação: como a face é exposta constantemente, caso uma base de faces seja roubada, essas informações não representam algum risco e não possibilitam um uso impróprio;
	\end{enumerate}

Umas das maiores dificuldades dos sistemas de reconhecimento é tratar a complexidade dos padrões visuais. Mesmo sabendo que todas as faces possuem padrões reconhecidos, como boca, olhos e nariz, elas também possuem variações únicas que devem ser utilizadas para determinar as características relevantes. Outra dificuldade encontrada em relação a essas características é que elas possuem uma larga variação estatística para serem consideradas únicas para cada indivíduo. O ideal seria que a variância inter-classe seja grande e a intra-classe pequena, pois assim imagens de diferentes faces geram os códigos mais diferentes possíveis, enquanto imagens de uma mesma face geram os códigos mais similares possíveis. Portanto, estabelecer uma representação que capture as características ideias é um difícil problema \cite{saocarlos}.

Entre os mais diferentes problemas encontrados nas tarefas envolvendo o reconhecimento facial, destacamos os mais comuns \cite{saocarlos}:

	\begin{itemize}
		\item iluminação;
		\item ângulos e poses;
		\item expressões;
		\item comésticos e acessórios;
		\item extração da face do contexto ou do fundo;
	\end{itemize}

No anos 70, os estudos do reconhecimento facial eram baseados sobre atributos faciais mensuráveis como olhos, nariz, sobrancelhas, bocas, entre outros. Porém, os recursos computacionais eram escassos e os algoritmos de extração de características eram ineficiêntes. Então, as pesquisas na área ressurgiram nos anos 90, inovando os métodos existentes \cite{hong, saocarlos}.

