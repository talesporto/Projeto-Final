\chapter{Resultados e Análises}
\label{cap:testes}

% - OK Achei o texto introdutório um pouco fraco em apresentar o objetivo do capítulo. 
% 	OK	Acho que vocês poderia focar primeiro que os testes foram para identificar a acurácia do sistema perante seus requisitos (Identificação e Localização). 
% OK		Apresentar o LAICO apenas como o local onde ocorreram os testes. 
% 		Apresentar cada ponto como uma medida/indicador do sistema desenvolvido. 
% 		OK No texto dos 4 testes foque nos objetivos e não no como. Tipo:
% 			Rastreamento: Testar a acurácia do rastreamento e identificação de objetos em situações do cotidiano.
% 			Reconhecimento: Testar a acurácia de identificação dos usuários perante a base de dados.

	Com intuito de identificar a acurácia e as limitações do Sistema TRUE perante seus requisitos (identificação, rastreamento e localização) foram feitos uma série de testes funcionais. Grande parte dos testes foram realizados no LAICO (\textbf{LA}boratório de Sistemas \textbf{I}ntegrados e \textbf{CO}ncorrente), um laboratório do Departamento de Ciência da Computação da Universidade de Brasília. 

	% O LAICO possui dimensões de $\displaystyle 7,67m$ x $\displaystyle 6,45m$ ilustrado pela Figura~\ref{fig:laico}.

	% \begin{figure}[htb]
	% 		\begin{center}
	% 			\includegraphics[scale=0.6]{figuras/4.ProblemaEProposta/laico.png}
	% 		\end{center}
	% 		\caption{Planta do \textit{SmartSpace} Laico.}
	% 		\label{fig:laico}
	% 	\end{figure}	

	Cada teste tinha como foco um das funcionalidades do Sistema TRUE:

	\begin{itemize}
		\item \textbf{Rastreamento}: testes funcionais realizados para avaliar a eficiência da detecção de novos usuários e que simulam diferentes situações diárias mostrando como o sistema se comporta quando ocorre oclusões parciais e totais de usuários e quando os mesmos interagem entre si ou com objetos no ambiente.

		\item \textbf{Reconhecimento}: testes realizados para avaliar a acurácia na identificação do usuários perante uma base de dados.

		% diferentes pessoas foram cadastradas no sistema e reconhecidas pelo mesmo. As diferentes identidades obtidas em cada reconhecimento foram inseridas em uma matriz de confusão para avaliar a acurácia do reconhecimento.

		\item \textbf{Localização}: testes realizados para avaliar a precisão do sistema ao estimar a posição dos usuários no ambiente.
	% um objeto foi colocado no ambiente em diferentes posições pre-determinadas e as coordenadas obtidas pelo sistema foram inseridas em gráficos e foram comparadas com as coordeanadas reais avaliando a precisão do sistema.

		\item \textbf{Integração com o middleware \textit{UbiquitOS}}: uma aplicação foi desenvolvida para testar a integração do Sistema TRUE com o middleware \textit{UbiquitOS}, testando os serviços disponíveis e os eventos gerados.

		% uma aplicação foi desenvolvida para o middleware \textit{UbiquitOS}. Ela consome alguns serviços e eventos produzidos pelo driver \textit{UserDriver} que são comparados com as informações reais.
	\end{itemize}

\section{Rastreamento dos Usuários}

	O rastreamento é fundamental para o funcionamento correto do Sistema TRUE uma
	vez que é responsável por rastrear os usuários no ambiente, determinar a sua localização
	física em relação ao sensor \textit{Kinect} e gerenciar suas identidades.
	Portanto, foi realizado uma série de testes funcionais para avaliar o modulo de
	rastreamento do Sistema TRUE.
		
	\begin{figure}[htb]
		\begin{center}
 			\includegraphics[width=0.4\textwidth]{figuras/4.ProblemaEProposta/laico-teste-deteccao.png}
 		\end{center}
 		\caption{Demostração do teste de detecção dentro do \textit{SmartSpace}
 		Laico.}
		\label{fig:laico-teste-deteccao}
	\end{figure}		
	
	Os primeiros testes realizados tiveram como objetivo verificar a eficiência da
	detecção de novos usuários no ambiente. Os testes foram feitos simulando a
	entrada de um usuário na cena e analisando o momento em que o mesmo era
	detectado. Esse teste foi executado no Laico e a posição do sensor e a
	trajetória do usuário é apresentado na Figura~\ref{fig:laico-teste-deteccao}. Em
	todos os testes o usuário era detectado antes mesmo de entrar na área de visão
	do sistema por completo, como mostrado na Figura~\ref{fig:testes_deteccao}.
	
	\begin{figure}[htb]
		\begin{center}
				\subfloat[] {
					\includegraphics[width=0.25\textwidth]{figuras/5.Testes/deteccao/1.png}}
				\subfloat[] {
					\includegraphics[width=0.25\textwidth]{figuras/5.Testes/deteccao/2.png}}
				\subfloat[] {
					\includegraphics[width=0.25\textwidth]{figuras/5.Testes/deteccao/3.png}}
				\subfloat[] {
					\includegraphics[width=0.25\textwidth]{figuras/5.Testes/deteccao/4.png}}
		\end{center}
		\caption{Momento em que um novo usuário foi detectado pelo Sistema TRUE.}
		\label{fig:testes_deteccao}
	\end{figure}
		
	Também foram realizados testes com o objetivo de verificar a oclusão no
	rastreamento. Foi testado o caso em que um usuário oculta propositalmente outro
	já rastreado. Nesta situação, caso o usuário rastreado continuasse oculto, o
	sistema o determinava como perdido. Por outro lado, nos casos de oclusão
	parcial, o sistema se mostrou robusto como pode ser observado na
	Figura~\ref{fig:testes_oclusao_sucesso}, onde o usuário apresentado na cor verde
	está parcialmente oculto pelo usuário na cor azul. Nos testes foi simulada
	também a situação de oclusão momentânea, ou seja, um usuário em movimento oculta
	outro por um curto período de tempo em razão da sua movimentação. Nestes casos,
	o sistema TRUE perde o usuário rastreado, mas rapidamente é capaz de detectá-lo
	novamente, como mostrado na sequência de imagens da
	Figura~\ref{fig:testes_oclusao}. A oclusão era um problema esperado uma vez que
	o Sistema TRUE utiliza somente um sensor \textit{Kinect} como dispositivo de
	entrada.
			
	\begin{figure}[htb]
		\begin{center}
			\includegraphics[width=0.75\textwidth]{figuras/5.Testes/oclusao/oclusao_corretamente.png}
		\end{center}
		\caption{Oclusão parcial de usuário rastreado.}
		\label{fig:testes_oclusao_sucesso}
	\end{figure}

	\begin{figure}[htb]
	\begin{center}
			\subfloat[] {
				\includegraphics[width=0.19\textwidth]{figuras/5.Testes/oclusao/1.png}}
			\subfloat[] {
				\includegraphics[width=0.19\textwidth]{figuras/5.Testes/oclusao/2.png}}
			\subfloat[] {
				\includegraphics[width=0.19\textwidth]{figuras/5.Testes/oclusao/3.png}}
			\subfloat[] {
				\label{fig:testes_oclusao_ocluso}
				\includegraphics[width=0.19\textwidth]{figuras/5.Testes/oclusao/4.png}}
			\subfloat[] {
				\includegraphics[width=0.19\textwidth]{figuras/5.Testes/oclusao/5.png}}
		\end{center}
		\caption{Oclusão de usuários.}
		\label{fig:testes_oclusao}
	\end{figure}

	Dado que o \textit{Kinect} tem um ângulo de 57º e foi comprovado em testes
	que o alcance maximo é de 4057 milímetros (Seção~\ref{sec:testes-localizacao})
	elaboramos o seguinte teste. A uma distância de aproximadamente 4 metros do
	sensor foram dispostos, ombro a ombro, o maximo de usuários sem que
	haja oclusão ou interferência. Neste teste conseguimos que até 5 usuários
	fossem rastreados como é possível ver na Figura~\ref{fig:max-pessoas}.

% 	Com relação a abrangência do campo de visão do Sistema TRUE, tendo em vista que
% 	ele utiliza o sensor \textit{Kinect} cujo campo de visão horizontal é de 57º e
% 	cujos testes mostraram que tem alcance maximo de 4,057 metros
% 	(Seção~\ref{sec:testes-localizacao}), é possível até uma distância de,
% 	aproximadamente, 4 metros do sensor, ter até 5 usuários no campo de visão sem
% 	que haja oclusão de pessoas (Figura~\ref{fig:max-pessoas}).
	
	% TODO: jogar isso aqui.
% 	Através deste teste, também foi possível obter a distância máxima e mínima que o usuário
% deve estar do Kinect para que o sistema consiga rastrea-lo e estimar sua localização. A distância
% mínima é de 48, 3cm e a máxima de 4, 057m.
	

	\begin{figure}[htb]
		\begin{center}
			\includegraphics[width=0.4\textwidth]{figuras/5.Testes/oclusao/max-pessoas.png}
		\end{center}
		\caption{Usuários posicionados lado a lado a uma distância de 4 metros do
		sensor \textit{Kinect}.}
		\label{fig:max-pessoas}
	\end{figure}
		
	Durante os testes realizados com rastreamento foi observado alguns problemas
	quando o usuário rastreado interagia com objetos do ambiente ou com outros
	usuários. Na grande maioria das vezes em que o usuário interagiu com objetos, o
	Sistema TRUE considerou o objeto como sendo parte do usuário, conforme
	exemplificado na Figura~\ref{fig:testes_relacionamento_com_objetos}. Entretando,
	esta situação não prejudicou a eficiência do sistema. Por outro lado, os
	problemas com interação entre usuários foram mais raros, porém tiveram impacto
	maior. Tais problemas consistem em ``interferências'' que ocorreram em algumas
	situações de contato entre dois ou mais usuários, conforme exemplifica a
	Figura~\ref{fig:testes_relacionamento_com_usuarios}.
	
	\begin{figure}[htb]
		\begin{center}
			\subfloat[] {
				\includegraphics[width=0.37\textwidth]{figuras/5.Testes/relacionamento_com_objetos/1.png}}
			\subfloat[] {
				\includegraphics[width=0.37\textwidth]{figuras/5.Testes/relacionamento_com_objetos/2.png}}
		\end{center}
		\caption{Usuário sendo rastreado em conjunto com o objeto que interage.}
		\label{fig:testes_relacionamento_com_objetos}
	\end{figure}
		
	\begin{figure}[htb]
		\begin{center}
			\subfloat[] {
				\includegraphics[width=0.32\textwidth]{figuras/5.Testes/relacionamento_com_pessoas/1.png}}
			\subfloat[] {
				\includegraphics[width=0.32\textwidth]{figuras/5.Testes/relacionamento_com_pessoas/2.png}}
			\subfloat[] {
				\includegraphics[width=0.32\textwidth]{figuras/5.Testes/relacionamento_com_pessoas/3.png}}
		\end{center}
		\caption{Usuários sofrendo interferência de outros ao seu redor.}
		\label{fig:testes_relacionamento_com_usuarios}
	\end{figure}

	Apesar dos problemas relatados, o rastreamento conseguiu, na maioria dos testes,
	atender às necessidades rastreando os diversos usuários no ambiente em suas
	atividades diárias, como mostrado na Figura~\ref{fig:varios-usuarios-ambiente}.

	\begin{figure}[htb]
		\begin{center}
			\includegraphics[scale=0.5]{figuras/5.Testes/oclusao/usuarios-rastreados.png}
		\end{center}
		\caption{Usuários do LAICO rastreados pelo Sistema TRUE.}
		\label{fig:varios-usuarios-ambiente}
	\end{figure}	

\section{Localização dos Usuários}

O Sistema TRUE obtém a localização dos usuários no ambiente por meio de coordenadas dos mesmos em relação ao \textit{Kinect}. Para saber o quanto essas coordenadas são confiáveis foram realizados alguns testes que resultaram em gráficos que comparam as coordenadas obtidas pelo sistema e as coordenadas reais.

As coordenadas $\displaystyle (x, y, z)$ obtidas pelo sistema são coordenas de um plano cartesiano de três dimensões em que o ponto $\displaystyle (0, 0, 0)$ corresponde a posição do \textit{Kinect}. Os valores das coordenadas que realmente são utilizadas para estimar a localização do usuário no ambiente são os valores nos eixos $\displaystyle x$ e $\displaystyle z$. Os valores do eixo $\displaystyle y$ correspondem somente a altura do centro de massa geométrico do usuário rastreado. Portanto, os testes desenvolvidos aferiram somente os valores obtidos pelo Sistema TRUE nos eixos $\displaystyle x$ e $\displaystyle z$.

\subsection{Teste dos valores no eixo $\displaystyle z$}
\subsection{Teste dos valores no eixo $\displaystyle x$}

\section{Identificação dos Usuários}
	 
	O reconhecimento facial do Sistema TRUE foi implementado utilizando imagens de cor como dados de entrada. Com isso, ele sofre um pouco com problemas já conhecidos e esperados como iluminação, ângulo, pose, expressões, cosméticos e acessórios. Portanto, para poder mensurar o quanto o reconhecimento do Sistema TRUE é confiável, foi criada uma matriz de confusão.

	Matriz de confusão é uma matriz em que cada coluna representa as instâncias de uma classe prevista, enquanto que cada linha representa as instâncias de uma classe real. Um dos benefícios dela é que é fácil ver se o sistema está confundindo duas classes, ou no caso, dois usuários.

	Para construir a matriz de confusão, foram cadastrados 11 usuários no Sistema TRUE. Cada usuários se posicionava em frente ao \textit{Kinect}que realizava o processo de reconhecimento 20 vezes. Os resultados obtidos, foram inseridos na matriz de confusão mostrada na Tabela~\ref{tab:matriz-confusao}.
	
	\begin{landscape}
	\begin{table}[htb]
		\begin{center}
			\caption{Matriz de confusão para apresentar os resultados obtidos.}
			\label{tab:matriz-confusao}
			\begin{tabular}{|c|c|c|c|c|c|c|c|c|c|c|c|c|}
				\hline \bf X & \bf Tales & \bf Danilo & \bf Ana & \bf Fabrício & \bf Lucas & \bf Bruno & \bf Ricardo & \bf Estevão & \bf Rafael & \bf Vinicius & \bf Pedro & \bf Desconhecido\\
				\hline \bf Tales 		& 95\% & 			& 		 & 			& & & 		 & & 		 & 			& 		 & 5\%\\
				\hline \bf Danilo 	& 		 & 75\% & 		 & 			& & & 		 & & 		 & 			& 5\%  &		\\
				\hline \bf Ana 			& 		 & 			& 95\% & 			& & & 		 & & 		 & 			& 		 & 5\%\\
				\hline \bf Fabrício & 20\% & 			& 		 & 50\% & & & 10\% & & 		 & 20\% & 		 &		\\
				\hline \bf Lucas 		& 		 & 			& 		 & 			& & & 		 & & 		 & 			& 		 &		\\
				\hline \bf Bruno 		& 		 & 			& 		 & 			& & & 		 & & 		 & 			& 		 &		\\
				\hline \bf Ricardo 	& 		 & 			& 10\% & 			& & & 85\% & & 		 & 			& 		 & 5\%\\
				\hline \bf Estevão 	& 		 & 			& 		 & 			& & & 		 & & 		 & 			& 		 &		\\
				\hline \bf Rafael 	& 		 & 			& 		 & 			& & & 		 & & 		 & 			& 		 &		\\
				\hline \bf Vinicius & 		 & 			& 5\%  & 			& & & 		 & & 5\% & 70\% & 10\% & 2\%\\
				\hline \bf Pedro 		& 		 & 			& 		 & 			& & & 		 & & 		 & 			& 		 &		\\
				\hline
			\end{tabular}
		\end{center}
	\end{table}
	\end{landscape}


\section{Integração com Middleware \textit{uOS}}

	A integração do Middleware \textit{uOS} com o Sistema TRUE consiste em um \textit{driver} desenvolvido para que ambos possam se comunicar. Este \textit{driver} foi nomeado de \textit{UserDriver} que foi apresentado na Seção~\ref{sec:modulo-integracao}. Com intuito de exemplificar a utilização do \textit{UserDriver} e testar a integração do Sistema TRUE com o middleware \textit{uOS} foi desenvolvido uma aplicação para o middleware chamada \textit{UserApp}. Esta aplicação registra um \textit{listener} para ``escutar'' os eventos do \textit{UserDriver} chamado \textit{UserListener}. 

	Como aplicação o \textit{UserApp} apenas se inscreve para receber os eventos gerados pelo \textit{UserDriver}. Já o \textit{UserListener}, como \textit{listener}, espera os eventos serem gerados e realiza duas análises: analisa o evento (novo usuário detectado, usuário perdido, ou reconhecimento realizado) e analisa quanto tempo cada usuário está parado no mesmo lugar.

		% \begin{figure}[hbt]
		% 	\begin{center}
		% 		\includegraphics[scale=0.6]{figuras/5.Testes/diagrama-classe-user-ap.png}
		% 	\end{center}
		% 	\caption{Diagrama de Classe da aplicação \textit{UserApp}.}
		% 	\label{fig:diagrama-userapp}
		% \end{figure}

		% \begin{figure}[hbt]
		% 	\begin{center}
		% 		\includegraphics[scale=0.6]{figuras/5.Testes/diagrama-classe-user-listener.png}
		% 	\end{center}
		% 	\caption{Diagrama de Classe do \textit{listener} \textit{UserListener}.}
		% 	\label{fig:diagrama-userlistener}
		% \end{figure}

	Basicamente, quando o \textit{UserListener} obtém os eventos do \textit{UserDriver}, ele envia mensagens pelo Twitter~\cite{twitter} para os usuários no ambiente, conforme o evento recebido. A Figura~\ref{fig:diagrama-tweet} mostra o fluxo básico de execução do \textit{listener} e as mensagens padrões para cada tipo de evento recebido. Para enviar as mensagens pelo Twitter foi utilizado a biblioteca \textit{twitter4j}~\cite{twitter4j}. 

	\begin{figure}[hbt]
		\begin{center}
			\includegraphics[scale=0.45]{figuras/5.Testes/diagrama-user-tweet.png}
		\end{center}
		\caption{Fluxo básico de execução do \textit{listener} \textit{UserListener}.}
		\label{fig:diagrama-tweet}
	\end{figure}

	Testes funcionais foram feitos com a aplicação mostrando que o \textit{driver} consegue
	obter os dados íntegros do Sistema TRUE e gerar os eventos de maneira praticamente
	instantânea. Algumas vezes as mensagens demoravam a chegar ao Twitter,
	geralmente nos horários de pico quando o Twitter operava próximo ao limite
	da sua capacidade. A Figura~\ref{fig:tweets} mostra as mensagens geradas
	pela aplicação em um teste funcional, onde Danilo, um usuário cadastrado,
	entra no ambiente senta em uma mesa com seu notebook permanecendo no mesmo
	lugar por mais de uma hora, e logo depois deixa o ambiente.

	\begin{figure}[hbt]
			\begin{center}
				\includegraphics[scale=0.6]{figuras/5.Testes/tweets.png}
			\end{center}
			\caption{Exemplo das mensagens enviadas pelo Twitter aos usuários no ambiente.}
			\label{fig:tweets}
		\end{figure}	











