\section{Identificação dos Usuários}
	 
	O reconhecimento facial do Sistema TRUE foi implementado utilizando imagens de cor como dados de entrada. Com isso, ele sofre um pouco com problemas já conhecidos e esperados como iluminação, ângulo, pose, expressões, cosméticos e acessórios. Portanto, para poder mensurar o quanto o reconhecimento do Sistema TRUE é confiável, foi criada uma matriz de confusão, representada na Tabela~\ref{tab:matriz-confusao}.

	Matriz de confusão é uma matriz em que cada coluna representa as instâncias de uma classe prevista, enquanto que cada linha representa as instâncias de uma classe real. Um dos benefícios dela é que é fácil ver se o sistema está confundindo duas classes, ou no caso, dois usuários.
	

	\begin{table}[H]
		\begin{center}
			\caption{Matriz de confusão para apresentar os resultados obtidos.}
			\begin{tabular}{|c|c|c|c|c|c|c|c|c|c|}
				\hline \bf X & \bf Tales & \bf Danilo & \bf Fabrício & \bf Ricardo & \bf
				Bruno & \bf Estevão & \bf Guto & \bf Tainá & \bf Tanyssa \\
				\hline \bf Tales & & & & & & & & & \\
				\hline \bf Danilo & & & & & & & & & \\
				\hline \bf Fabrício & & & & & & & & & \\
				\hline \bf Ricardo & & & & & & & & & \\
				\hline \bf Bruno & & & & & & & & & \\
				\hline \bf Estevão & & & & & & & & & \\
				\hline \bf Guto & & & & & & & & & \\
				\hline \bf Tainá & & & & & & & & & \\
				\hline \bf Tanyssa & & & & & & & & & \\
				\hline
			\end{tabular}
		\end{center}
		\label{tab:matriz-confusao}
	\end{table}
