\section{Identificação dos Usuários}
	
	Os usuários são identificados através do metodo conhecido como
	Eigenfaces~\ref{sec:reconhecimento}. Os usuários são identificados por suas
	labels de cadastro e a cada reconhecimento é associado a essa label uma
	confiança. Para validar a confiabilidade da identificação será usado a matriz
	de confusão para representar os dados dos testes.
	
	\begin{table}[H]
		\begin{center}
			\caption{Matriz de confusão para apresentar os resultados obtidos.}
			\begin{tabular}{|c|c|c|c|c|c|c|c|c|c|}
				\hline \bf X & \bf Tales & \bf Danilo & \bf Fabrício & \bf Ricardo & \bf
				Bruno & \bf Estevão & \bf Guto & \bf Tainá & \bf Tanyssa \\
				\hline \bf Tales & & & & & & & & & \\
				\hline \bf Danilo & & & & & & & & & \\
				\hline \bf Fabrício & & & & & & & & & \\
				\hline \bf Ricardo & & & & & & & & & \\
				\hline \bf Bruno & & & & & & & & & \\
				\hline \bf Estevão & & & & & & & & & \\
				\hline \bf Guto & & & & & & & & & \\
				\hline \bf Tainá & & & & & & & & & \\
				\hline \bf Tanyssa & & & & & & & & & \\
				\hline
			\end{tabular}
		\end{center}
		\label{tab:tabelaRequisitosTeoricos}
	\end{table}
