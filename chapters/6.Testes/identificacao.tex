\section{Identificação dos Usuários}
	 
	O reconhecimento facial do Sistema TRUE foi implementado utilizando imagens de cor como dados de entrada. Portanto, ele tem influência de diversos fatores como iluminação, ângulo, pose, expressões, cosméticos e acessórios. Portanto, para análise dos resultados foi utilizada como ferramenta uma matriz de confusão construída a partir dos dados do sistema.

	Tal técnica consiste em uma matriz em que cada coluna representa as instâncias de uma classe prevista, enquanto que cada linha representa as instâncias de uma classe real. Como benefício da técnica temos a fácil visualização da confiança obtida no reconhecimento, isso pode ser observado através dos valores obtidos na diagonal principal que representa a porcentagem de acerto para cada usuário.

	Nosso cenário foi construído com 11 usuários cadastrados préviamente além das 40 pessoas presentes no banco de faces da Universidade de Cambridge~\cite{cambridgeFaceDb}. Cada usuário se posicionava em frente ao \textit{Kinect}, com o rosto posicionado de maneira frontal em relação ao mesmo, que realizava o processo de reconhecimento 20 vezes. As identidades obtidas pelo Sistema TRUE foram inseridas na matriz de confusão mostrada na Tabela~\ref{tab:matriz-confusao}.
	
	\begin{landscape}
	\begin{table}[htb]
		\begin{center}
			\caption{Matriz de confusão para apresentar os resultados obtidos.}
			\label{tab:matriz-confusao}
			\begin{tabular}{|c|c|c|c|c|c|c|c|c|c|c|c|c|}
				\hline  & \bf Tales & \bf Danilo & \bf Ana & \bf Fabrício & \bf Lucas & \bf Bruno & \bf Ricardo & \bf Estevão & \bf Rafael & \bf Vinicius & \bf Pedro & \bf Desconhecido\\
				\hline \bf Tales 		& 95\% & 			& 		 & 			&   	 & 			& 		 & 			& 		 & 			& 		 & 5\%	\\
				\hline \bf Danilo 	& 		 & 75\% & 		 & 			&   	 & 			& 		 & 			& 		 & 			& 25\% &		 	\\
				\hline \bf Ana 			& 		 & 			& 95\% & 			&   	 & 			& 		 & 			& 		 & 			& 		 & 5\%  \\
				\hline \bf Fabrício & 20\% & 			& 		 & 50\% &      & 			& 10\% & 			&  	   & 20\% & 		 &		  \\
				\hline \bf Lucas 		& 		 & 			& 		 & 			& 55\% & 			& 		 & 			& 		 & 			& 20\% & 25\% \\
				\hline \bf Bruno 		& 5\%	 & 			& 		 & 			& 		 & 70\% & 		 & 			& 10\% & 	5\%	& 		 & 10\%	\\
				\hline \bf Ricardo 	& 		 & 			& 10\% & 			& 		 & 			& 85\% & 			& 		 & 			& 		 & 5\%  \\
				\hline \bf Estevão 	& 		 & 			& 		 & 			& 		 & 			& 		 & 70\% & 		 & 			& 		 & 30\% \\
				\hline \bf Rafael 	& 10\% & 	5\%	& 		 & 			& 10\% & 			& 		 & 			& 45\% & 20\% & 		 & 10\% \\
				\hline \bf Vinicius & 		 & 			& 5\%  & 			& 		 & 			& 		 & 			& 5\%  & 70\% & 10\% & 10\% \\
				\hline \bf Pedro 		& 		 & 			& 		 & 			& 		 & 			& 		 & 			& 		 & 			& 100\%&		  \\
				\hline
				% \hline \bf Média de Acertos & \multicolumn{12}{|l|}{73,63\%} \\
				% \hline
			\end{tabular}
		\end{center}
	\end{table}
	\end{landscape}

	Como visto na matriz, os maiores valores, como previsto, estão na diagonal principal. Alguns resultados obtidos foram muito bons e outros ficaram aquém do esperado. Estes últimos foram principalmente causados por problemas já esperados como pose e expressões faciais. A iluminação não foi um problema, pois no ambiente de teste não existe influência da iluminação externa e a interna é controlada. Para se ter uma melhor interpretação dos valores obtidos, foram retiradas três taxas desta matriz:

	\begin{enumerate}
		\item \textbf{Verdadeiro Positivo}: quando o sistema identifica o usuário de maneira correta.
		\item \textbf{Verdadeiro Negativo}: quando o sistema identifica o usuário de maneira errada.
		\item \textbf{Falso Negativo}: quando o sistema não identifica o usuário cadastrado.
	\end{enumerate}

	Tais taxas foram inseridas na Tabela~\ref{tab:taxas}. A taxa de acerto (verdadeito positivo) foi menor do que se esperava. Portanto, o teste foi realizado novamente utilizando menos usuários, 6 usuários além dos 40 presentes no banco de faces da Universidade de Cambridge~\cite{cambridgeFaceDb}. Porém o cadastro destes usuários foi feito de maneira diferente. Ao invés de se obter 10 fotos das faces posicionadas de maneira frontal em relação ao sensor, foram obtidas 100 imagens das faces em diferentes ângulos e posições. Os resultados obtidos, foram inseridos em uma segunda matriz de confusão (Tabela~\ref{tab:matriz-confusao2}).

	\begin{table}[htb]
		\begin{center}
			\caption{Taxas obtidas da Tabela~\ref{tab:matriz-confusao}.}
			\label{tab:taxas}
			\begin{tabular}{|l|c|}
				\hline \bf Verdadeiro Positivo & 73,63\% \\
				\hline \bf Verdadeiro Negativo & 17,27\% \\
				\hline \bf Falso Negativo & 9,10\% \\
				\hline
			\end{tabular}
		\end{center}
	\end{table}


	% \begin{landscape}
	\begin{table}[htb]
		\begin{center}
			\caption{Matriz de confusão para apresentar os resultados obtidos.}
			\label{tab:matriz-confusao2}
			\begin{tabular}{|c|c|c|c|c|c|c|c|c|}
				\hline  & \bf Tales & \bf Danilo & \bf Ana & \bf Fabrício & \bf Lucas & \bf Bruno &  \bf Desconhecido\\
				\hline \bf Tales 		& 90\% & 			& 		 & 			&   	 & 			& 10\%		\\
				\hline \bf Danilo 	& 		 & 100\%& 		 & 			&   	 & 			& 		 		\\
				\hline \bf Ana 			& 		 & 			& 100\%& 			&   	 & 			& 		    \\
				\hline \bf Fabrício & 		 & 			& 		 &      &      & 			&      		\\
				\hline \bf Lucas 		& 		 & 			& 		 & 			& 85\% & 			& 15\%		\\
				\hline \bf Bruno 		& 		 & 			& 		 & 			& 		 & 100\%& 		  	\\
				\hline
			\end{tabular}
		\end{center}
	\end{table}
	% \end{landscape}


	

