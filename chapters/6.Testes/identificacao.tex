\section{Identificação dos Usuários}
	 
	O reconhecimento facial do Sistema TRUE foi implementado utilizando imagens de cor como dados de entrada. Portanto, ele tem influência de diversos fatores como iluminação, ângulo, pose, expressões, cosméticos e acessórios. Portanto, para análise dos resultados foi utilizada como ferramenta uma matriz de confusão construída a partir dos dados do sistema.

	Tal técnica consiste em uma matriz em que cada coluna representa as instâncias de uma classe prevista, enquanto que cada linha representa as instâncias de uma classe real. Como benefício da técnica temos a fácil visualização da confiança obtida no reconhecimento, isso pode ser observado através dos valores obtidos na diagonal principal que representa a porcentagem de acerto para cada usuário.

	Nosso cenário foi construído com 11 usuários cadastrados préviamente além das 40 pessoas presentes no banco de faces da
	Universidade de Cambridge~\cite{cambridgeFaceDb}. Cada usuário se posicionava
	em frente ao \textit{Kinect}, com o rosto posicionado de maneira frontal em
	relação ao mesmo, que realizava o processo de reconhecimento 20 vezes. As
	identidades obtidas pelo Sistema TRUE foram inseridas na matriz de confusão
	mostrada na Tabela~\ref{tab:matriz-confusao}.
	
	% \begin{enumerate}
	%   \bf \item Tales \label{user:tales}
	%   \bf \item Danilo \label{user:danilo}
	%   \bf \item Ana \label{user:ana}
	%   \bf \item Fabricio \label{user:fabricio}
	%   \bf \item Lucas \label{user:lucas}
	%   \bf \item Bruno \label{user:bruno}
	%   \bf \item Ricardo \label{user:ricardo}
	%   \bf \item Estevão \label{user:estevao}
	%   \bf \item Rafael \label{user:rafael}
	%   \bf \item Vinicius \label{user:vinicius}
	%   \bf \item Pedro \label{user:pedro}
	%   \label{list:usuarios}
	% \end{enumerate}
	
	% \begin{landscape}
	\begin{table}[htb]
		\begin{center}
			\caption{Matriz de confusão para apresentar os resultados obtidos.}
			\label{tab:matriz-confusao}
			\begin{tabular}{|c|c|c|c|c|c|c|c|c|c|c|c|c|}
				% \hline  & \bf \ref{user:tales} & \bf \ref{user:danilo} & \bf
				% \ref{user:ana} & \bf \ref{user:fabricio} & \bf \ref{user:lucas} & \bf
				% \ref{user:bruno} & \bf \ref{user:ricardo} & \bf \ref{user:estevao} & \bf \ref{user:rafael} &
				% \bf \ref{user:vinicius} & \bf \ref{user:pedro} & \bf Desconhecido\\ 

				\hline  & \bf \begin{sideways}Tales\end{sideways} & \bf \begin{sideways}Danilo\end{sideways} & \bf \begin{sideways}Ana\end{sideways} & \bf \begin{sideways}Fabricio\end{sideways} & \bf \begin{sideways}Lucas\end{sideways} & \bf \begin{sideways}Bruno\end{sideways} & \bf \begin{sideways}Ricardo\end{sideways} & \bf \begin{sideways}Estevao\end{sideways} & \bf \begin{sideways}Rafael\end{sideways} &
				\bf \begin{sideways}Vinicius\end{sideways} & \bf \begin{sideways}Pedro\end{sideways} & \bf \begin{sideways}Desconhecido\end{sideways}\\ 
				
				\hline \bf Tales 		& 95\% & 			& 		 & 			&   	 & 			& 		 & 			& 		 & 			& 		 & 5\%	\\ 
				\hline \bf Danilo 	& 		 & 75\% & 		 & 			&   	 & 			& 		 & 			& 		 & 			& 25\% &		 	\\
				\hline \bf Ana 			& 		 & 			& 95\% & 			&   	 & 			& 		 & 			& 		 & 			& 		 & 5\%  \\
				\hline \bf Fabricio & 20\% & 			& 		 & 50\% &      & 			& 10\% & 			&  	   & 20\% & 		 &		  \\
				\hline \bf Lucas 		& 		 & 			& 		 & 			& 55\% & 			& 		 & 			& 		 & 			& 20\% & 25\% \\
				\hline \bf Bruno 		& 5\%	 & 			& 		 & 			& 		 & 70\% & 		 & 			& 10\% & 	5\%	& 		 & 10\%	\\
				\hline \bf Ricardo 	& 		 & 			& 10\% & 			& 		 & 			& 85\% & 			& 		 & 			& 		 & 5\%  \\
				\hline \bf Estevao 	& 		 & 			& 		 & 			& 		 & 			& 		 & 70\% & 		 & 			& 		 & 30\% \\
				\hline \bf Rafael 	& 10\% & 	5\%	& 		 & 			& 10\% & 			& 		 & 			& 45\% & 20\% & 		 & 10\% \\
				\hline \bf Vinicius & 		 & 			& 5\%  & 			& 		 & 			& 		 & 			& 5\%  & 70\% & 10\% & 10\% \\
				\hline \bf Pedro 		& 		 & 			& 		 & 			& 		 & 			& 		 & 			& 		 & 			& 100\%&		  \\
				\hline
				% \hline \bf Média de Acertos & \multicolumn{12}{|l|}{73,63\%} \\
				% \hline
			\end{tabular}
		\end{center}
	\end{table}
	% \end{landscape}

	Como visto na matriz, os maiores valores, como previsto, estão na diagonal
	principal. Alguns resultados obtidos foram satisfatórios já outros ficaram
	aquém do esperado. Estes últimos foram principalmente causados por problemas como pose e expressões faciais. A iluminação não foi um problema,
	pois no ambiente de teste não existe influência da iluminação externa e a
	interna é controlada. Para se ter uma melhor interpretação dos valores obtidos,
	foram retiradas três taxas desta matriz:

	\begin{enumerate}
		\item \textbf{Verdadeiro Positivo}: quando o sistema identifica o usuário de maneira correta.
		\item \textbf{Verdadeiro Negativo}: quando o sistema identifica o usuário de maneira errada.
		\item \textbf{Falso Negativo}: quando o sistema não identifica o usuário cadastrado.
	\end{enumerate}

	Tais taxas foram inseridas na Tabela~\ref{tab:taxas}. A taxa de acerto
	(verdadeito positivo) foi menor do que se esperava. Portanto, o teste foi
	realizado novamente utilizando menos usuários, 6 usuários além dos 40
	presentes no banco de faces da Universidade de
	Cambridge~\cite{cambridgeFaceDb}. Porém o cadastro destes usuários foi feito
	de maneira diferente. Ao invés de se obter 10 fotos das faces posicionadas de
	maneira frontal em relação ao sensor, foram obtidas 100 imagens das faces em
	diferentes ângulos, posições e expressões faciais. Os resultados obtidos, foram inseridos em uma
	segunda matriz de confusão (Tabela~\ref{tab:matriz-confusao2}).

	\begin{table}[htb]
		\begin{center}
			\caption{Taxas obtidas da Tabela~\ref{tab:matriz-confusao}.}
			\label{tab:taxas}
			\begin{tabular}{|l|c|}
				\hline \bf Verdadeiro Positivo & 73,63\% \\
				\hline \bf Verdadeiro Negativo & 17,27\% \\
				\hline \bf Falso Negativo & 9,10\% \\
				\hline
			\end{tabular}
		\end{center}
	\end{table}


	% \begin{table}[htb]
	% 	\begin{center}
	% 		\caption{Matriz de confusão para apresentar os resultados obtidos.}
	% 		\label{tab:matriz-confusao2}
	% 		  \begin{tabular}{|c|c|c|c|c|c|c|c|c|}
	% 			% \hline  & \bf \ref{user:tales} & \bf \ref{user:danilo} & \bf
	% 			% \ref{user:ana} & \bf \ref{user:fabricio} & \bf \ref{user:lucas} & \bf
	% 			% \ref{user:bruno} &  \bf Desconhecido\\

	% 			\hline  & \bf Tales & \bf Danilo & \bf Ana & \bf Fabricio & \bf Lucas & \bf Bruno &  \bf Desconhecido\\
				 
	% 			\hline \bf Tales 		& 90\% & 			& 		 & 			&   	 & 			& 10\%		\\
	% 			\hline \bf Danilo 	& 		 & 100\%& 		 & 			&   	 & 			& 		 		\\
	% 			\hline \bf Ana 			& 		 & 			& 100\%& 			&   	 & 			& 		    \\
	% 			\hline \bf Fabricio & 		 & 			& 		 &100\%      &      & 			&      		\\
	% 			\hline \bf Lucas 		& 		 & 			& 		 & 			& 85\% & 			& 15\%		\\
	% 			\hline \bf Bruno 		& 		 & 			& 		 & 			& 		 & 100\%& 		  	\\
	% 			\hline
	% 		\end{tabular}
	% 	\end{center}
	% \end{table}

	\begin{table}[htb]
		\begin{center}
			\caption{Matriz de confusão para apresentar os resultados obtidos.}
			\label{tab:matriz-confusao2}
			  \begin{tabular}{|c|c|c|c|c|c|c|c|c|c|c|c|c|}
				% \hline  & \bf \ref{user:tales} & \bf \ref{user:danilo} & \bf
				% \ref{user:ana} & \bf \ref{user:fabricio} & \bf \ref{user:lucas} & \bf
				% \ref{user:bruno} &  \bf Desconhecido\\

				\hline  & \bf \begin{sideways}Tales\end{sideways} & \bf \begin{sideways}Danilo\end{sideways} & \bf \begin{sideways}Ana\end{sideways} & \bf \begin{sideways}Fabricio\end{sideways} & \bf \begin{sideways}Lucas\end{sideways} & \bf \begin{sideways}Bruno\end{sideways} & \bf \begin{sideways}Carla\end{sideways} & \bf \begin{sideways}Marcela\end{sideways} & \bf \begin{sideways}Caio\end{sideways} & \bf \begin{sideways}Marcelo\end{sideways} & \bf \begin{sideways}Desconhecido\end{sideways}\\
				 
				\hline \bf Tales 			&100\% & 			& 		 & 			&   	 & 			& 	& 		&	 		& 		& 		\\
				\hline \bf Danilo 		& 		 &100\%	& 		 & 			&   	 & 			& 	& 		& 		& 		& 		 		\\
				\hline \bf Ana 				& 		 & 			& 		 & 			&   	 & 			& 	& 		& 		& 		& 		    \\
				\hline \bf Fabricio 	& 		 & 			& 		 &			&      & 			& 	& 		& 		& 		&      		\\
				\hline \bf Lucas 			& 		 & 			& 		 & 			& 85\% & 			& 	& 		& 		& 		& 15\%		\\
				\hline \bf Bruno 			& 		 & 			& 		 & 			& 		 & 75\% & 	& 		&20\% & 5\%	& 		  	\\
				\hline \bf Carla 			& 		 & 			& 		 & 			& 		 & 			& 	&		  & 	  & 		& 		  	\\
				\hline \bf Marcela 		& 		 & 			& 		 & 			& 		 &      & 	&65\% &10\% & 		& 25\%		  	\\
				\hline \bf Caio 		  & 		 & 			& 		 & 			& 	5\%& 	15\%& 	& 		&80\%	&	 		& 		  	\\
				\hline \bf Marcelo 		& 		 & 			& 		 & 			& 		 & 			& 	& 		& 		& 90\%& 10\%		  	\\
				\hline
			\end{tabular}
		\end{center}
	\end{table}

	\begin{table}[htb]
		\begin{center}
			\caption{Taxas obtidas da Tabela~\ref{tab:matriz-confusao2}.}
			\label{tab:taxas2}
			\begin{tabular}{|l|c|}
				\hline \bf Verdadeiro Positivo & 95,83\% \\
				\hline \bf Verdadeiro Negativo & 0\% \\
				\hline \bf Falso Negativo & 4,17\% \\
				\hline
			\end{tabular}
		\end{center}
	\end{table}

	Ao analisar os dados da Tabela~\ref{tab:matriz-confusao2} é fácil observar que os resultados melhoraram significamente. Essa melhora é confirmada pelas taxas retiradas desta tabela e inserida na Tabela~\ref{tab:taxas2}. Houve um bom aumento na taxa de Verdadeiro Positivo de 73,63\% para 95,83\% e redução na taxa de Verdadeiro Negativo de 17,27\% para 0\%. Tal melhora aconteceu devido a mudança na base de dados tornando o Sistema TRUE mais robusto com as variações de poses, ãngulos e expressões faciais.

	% Para esta matriz obtemos as taxas apresentadas na Tabela~\ref{tab:taxas2}. É
	% possível observar que o método de cadastro influi bastante no resultado final.
	% Também é possível inferir que quanto maior a variância das imagens para um
	% mesmo usuário melhor o seu reconhecimento.

