\section{Identificação dos Usuários}
	 
	O reconhecimento facial do Sistema TRUE foi implementado utilizando imagens de cor como dados de entrada. Com isso, ele sofre um pouco com problemas já conhecidos e esperados como iluminação, ângulo, pose, expressões, cosméticos e acessórios. Portanto, para poder mensurar o quanto o reconhecimento do Sistema TRUE é confiável, foi criada uma matriz de confusão.

	Matriz de confusão é uma matriz em que cada coluna representa as instâncias de uma classe prevista, enquanto que cada linha representa as instâncias de uma classe real. Um dos benefícios dela é que é fácil ver se o sistema está confundindo duas classes, ou no caso, dois usuários.

	Para construir a matriz de confusão, foram cadastrados 11 usuários no Sistema TRUE. Cada usuários se posicionava em frente ao \textit{Kinect}que realizava o processo de reconhecimento 20 vezes. Os resultados obtidos, foram inseridos na matriz de confusão mostrada na Tabela~\ref{tab:matriz-confusao}.
	
	\begin{landscape}
	\begin{table}[htb]
		\begin{center}
			\caption{Matriz de confusão para apresentar os resultados obtidos.}
			\label{tab:matriz-confusao}
			\begin{tabular}{|c|c|c|c|c|c|c|c|c|c|c|c|c|}
				\hline \bf X & \bf Tales & \bf Danilo & \bf Ana & \bf Fabrício & \bf Lucas & \bf Bruno & \bf Ricardo & \bf Estevão & \bf Rafael & \bf Vinicius & \bf Pedro & \bf Desconhecido\\
				\hline \bf Tales 		& 95\% & 			& 		 & 			& & & 		 & & 		 & 			& 		 & 5\%\\
				\hline \bf Danilo 	& 		 & 75\% & 		 & 			& & & 		 & & 		 & 			& 5\%  &		\\
				\hline \bf Ana 			& 		 & 			& 95\% & 			& & & 		 & & 		 & 			& 		 & 5\%\\
				\hline \bf Fabrício & 20\% & 			& 		 & 50\% & & & 10\% & & 		 & 20\% & 		 &		\\
				\hline \bf Lucas 		& 		 & 			& 		 & 			& & & 		 & & 		 & 			& 		 &		\\
				\hline \bf Bruno 		& 		 & 			& 		 & 			& & & 		 & & 		 & 			& 		 &		\\
				\hline \bf Ricardo 	& 		 & 			& 10\% & 			& & & 85\% & & 		 & 			& 		 & 5\%\\
				\hline \bf Estevão 	& 		 & 			& 		 & 			& & & 		 & & 		 & 			& 		 &		\\
				\hline \bf Rafael 	& 		 & 			& 		 & 			& & & 		 & & 		 & 			& 		 &		\\
				\hline \bf Vinicius & 		 & 			& 5\%  & 			& & & 		 & & 5\% & 70\% & 10\% & 2\%\\
				\hline \bf Pedro 		& 		 & 			& 		 & 			& & & 		 & & 		 & 			& 		 &		\\
				\hline
			\end{tabular}
		\end{center}
	\end{table}
	\end{landscape}
