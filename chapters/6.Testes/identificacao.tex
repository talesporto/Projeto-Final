\section{Identificação dos Usuários}
	 
	O reconhecimento facial do Sistema TRUE foi implementado utilizando imagens de cor como dados de entrada. Com isso, ele sofre um pouco com problemas já conhecidos e esperados como iluminação, ângulo, pose, expressões, cosméticos e acessórios. Portanto, para poder mensurar o quanto o reconhecimento do Sistema TRUE é confiável, foi criada uma matriz de confusão.

	Matriz de confusão é uma matriz em que cada coluna representa as instâncias de uma classe prevista, enquanto que cada linha representa as instâncias de uma classe real. Um dos benefícios dela é que é fácil ver se o sistema está confundindo duas classes, ou no caso, dois usuários.

	Para construir a matriz de confusão, foram cadastrados 11 usuários no Sistema
	TRUE. Cada usuário se posicionava em frente ao \textit{Kinect} que realizava
	o processo de reconhecimento 20 vezes. Os resultados obtidos, foram inseridos
	na matriz de confusão mostrada na Tabela~\ref{tab:matriz-confusao}.
	
	\begin{landscape}
	\begin{table}[htb]
		\begin{center}
			\caption{Matriz de confusão para apresentar os resultados obtidos.}
			\label{tab:matriz-confusao}
			\begin{tabular}{|c|c|c|c|c|c|c|c|c|c|c|c|c|}
				\hline  & \bf Tales & \bf Danilo & \bf Ana & \bf Fabrício & \bf Lucas & \bf Bruno & \bf Ricardo & \bf Estevão & \bf Rafael & \bf Vinicius & \bf Pedro & \bf Desconhecido\\
				\hline \bf Tales 		& 95\% & 			& 		 & 			&   	 & 			& 		 & 			& 		 & 			& 		 & 5\%	\\
				\hline \bf Danilo 	& 		 & 75\% & 		 & 			&   	 & 			& 		 & 			& 		 & 			& 25\% &		 	\\
				\hline \bf Ana 			& 		 & 			& 95\% & 			&   	 & 			& 		 & 			& 		 & 			& 		 & 5\%  \\
				\hline \bf Fabrício & 20\% & 			& 		 & 50\% &      & 			& 10\% & 			&  	   & 20\% & 		 &		  \\
				\hline \bf Lucas 		& 		 & 			& 		 & 			& 55\% & 			& 		 & 			& 		 & 			& 20\% & 25\% \\
				\hline \bf Bruno 		& 5\%	 & 			& 		 & 			& 		 & 70\% & 		 & 			& 10\% & 	5\%	& 		 & 10\%	\\
				\hline \bf Ricardo 	& 		 & 			& 10\% & 			& 		 & 			& 85\% & 			& 		 & 			& 		 & 5\%  \\
				\hline \bf Estevão 	& 		 & 			& 		 & 			& 		 & 			& 		 & 70\% & 		 & 			& 		 & 30\% \\
				\hline \bf Rafael 	& 10\% & 	5\%	& 		 & 			& 10\% & 			& 		 & 			& 45\% & 20\% & 		 & 10\% \\
				\hline \bf Vinicius & 		 & 			& 5\%  & 			& 		 & 			& 		 & 			& 5\%  & 70\% & 10\% & 10\% \\
				\hline \bf Pedro 		& 		 & 			& 		 & 			& 		 & 			& 		 & 			& 		 & 			& 100\%&		  \\
				\hline
				% \hline \bf Média de Acertos & \multicolumn{12}{|l|}{73,63\%} \\
				% \hline
			\end{tabular}
		\end{center}
	\end{table}
	\end{landscape}

	Como visto na matriz, os maiores valores, como previsto, estão na diagonal principal. Alguns resultados obtidos foram muito bons e outros ficaram aquém do esperado. Estes últimos foram principalmente causados por problemas já esperados como pose e expressões faciais. A iluminação não foi um problema, pois no ambiente de teste não existe influência da iluminação externa e a interna é controlada. Para se ter uma melhor interpretação dos valores obtidos, foram retiradas três taxas desta matriz:

	\begin{enumerate}
		\item \textbf{Verdadeiro Positivo}: quando o sistema identifica o usuário de maneira correta.
		\item \textbf{Verdadeiro Negativo}: quando o sistema identifica o usuário de maneira errada.
		\item \textbf{Falso Negativo}: quando o sistema não identifica o usuário cadastrado.
	\end{enumerate}

	Tais taxas foram inseridas na Tabela~\ref{tab:taxas}. A taxa de acerto (verdadeito positivo) foi menor do que se esperava. Então, esse mesmo teste foi realizado novamente, porém os usuários foram recadastrados utilizando 100 imagens com posições váriadas do rosto para cada usuário. Os dados obtidos foram inseridos na Tabela~\ref{tab:matriz-confusao2}.

	\begin{landscape}
	\begin{table}[htb]
		\begin{center}
			\caption{Matriz de confusão para apresentar os resultados obtidos.}
			\label{tab:matriz-confusao}
			\begin{tabular}{|c|c|c|c|c|c|c|c|c|c|c|c|c|}
				\hline  & \bf Tales & \bf Danilo & \bf Ana & \bf Fabrício & \bf Lucas & \bf Bruno & \bf Ricardo & \bf Estevão & \bf Rafael & \bf Vinicius & \bf Pedro & \bf Desconhecido\\
				\hline \bf Tales 		& 		 & 			& 		 & 			&   	 & 			& 		 & 			& 		 & 			& 		 & 	\\
				\hline \bf Danilo 	& 		 & 			& 		 & 			&   	 & 			& 		 & 			& 		 & 			& 		 &		 	\\
				\hline \bf Ana 			& 		 & 			& 		 & 			&   	 & 			& 		 & 			& 		 & 			& 		 &   \\
				\hline \bf Fabrício & 		 & 			& 		 &      &      & 			&      & 			&  	   & 		  & 		 &		  \\
				\hline \bf Lucas 		& 		 & 			& 		 & 			&      & 			& 		 & 			& 		 & 			&      &  \\
				\hline \bf Bruno 		& 		 & 			& 		 & 			& 		 &      & 		 & 			& 		 & 	  	& 		 & 	\\
				\hline \bf Ricardo 	& 		 & 			& 		 & 			& 		 & 			&      & 			& 		 & 			& 		 &   \\
				\hline \bf Estevão 	& 		 & 			& 		 & 			& 		 & 			& 		 &      & 		 & 			& 		 &  \\
				\hline \bf Rafael 	& 	 	 &	  	& 		 & 			&      & 			& 		 & 			&      & 		  & 		 &  \\
				\hline \bf Vinicius & 		 & 			& 		 & 			& 		 & 			& 		 & 			& 		 & 			& 		 &  \\
				\hline \bf Pedro 		& 		 & 			& 		 & 			& 		 & 			& 		 & 			& 		 & 			&      &		  \\
				\hline
			\end{tabular}
		\end{center}
	\end{table}
	\end{landscape}


	\begin{table}[htb]
		\begin{center}
			\caption{Taxas obtidas da Tabela~\ref{tab:matriz-confusao}.}
			\label{tab:taxas}
			\begin{tabular}{|l|c|}
				\hline \bf Verdadeiro Positivo & 73,63\% \\
				\hline \bf Verdadeiro Negativo & 17,27\% \\
				\hline \bf Falso Negativo & 9,10\% \\
				\hline
			\end{tabular}
		\end{center}
	\end{table}

