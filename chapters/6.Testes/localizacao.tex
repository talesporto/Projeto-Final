\section{Localização dos Usuários}

O Sistema TRUE obtém a localização dos usuários no ambiente por meio de coordenadas dos mesmos em relação ao \textit{Kinect}. Para saber o quanto essas coordenadas são confiáveis foram realizados alguns testes que resultaram em gráficos que comparam as coordenadas obtidas pelo sistema e as coordenadas reais.

As coordenadas $\displaystyle (x, y, z)$ obtidas pelo sistema são coordenas de um plano cartesiano de três dimensões em que o ponto $\displaystyle (0, 0, 0)$ corresponde a posição do \textit{Kinect}. Os valores das coordenadas que realmente são utilizadas para estimar a localização do usuário no ambiente são os valores nos eixos $\displaystyle x$ e $\displaystyle z$. Os valores do eixo $\displaystyle y$ correspondem somente a altura do centro de massa geométrico do usuário rastreado. Portanto, os testes desenvolvidos aferiram somente os valores obtidos pelo Sistema TRUE nos eixos $\displaystyle x$ e $\displaystyle z$.

\subsection{Teste dos valores no eixo $\displaystyle z$}
\subsection{Teste dos valores no eixo $\displaystyle x$}