\section{Módulo de Rastreamento}

	O Módulo de Rastreamento será responsável por rastrear os usuários no \textit{SmartSpace}, determinar a localização física de cada um em relação ao \textit{Kinect} e gerenciar suas identidades.

	Para realizar rastreamento e localização dos usuários será utilizado a implementação existente na biblioteca \textit{OpenNI} (\textit{Open Natural Interaction}). Trata-se de um \textit{framework} que define \textit{APIs} para o desenvolvimento de aplicações utilizando interação natural. Utilizando as imagens de profundidade, a detecção e o rastreamento dos usuários será feita utilizando subtração de fundo~\ref{sec:deteccao-objeto} e a representação dos usuários será feita por silhuetas~\ref{sec:representacao-objeto}. 

	O rastreamento e a localização serão feitas utilizando as imagens de profundidades providas pelo \textit{Kinect} obtidas utilizando o método de Luz Estruturada descrito na seção~\ref{sec:luz-estruturada}, tornando-os não susceptíveis as variações nas condições de iluminação. Essas imagens de profundidade nada mais são que \textit{depth maps} (mapas de profundidade), em que cada pixel da imagem contém o valor estimado da distância em relação ao sensor. O \textit{Kinect} fornece esses dados a uma taxa de $\displaystyle 30 fps$ (\textit{frames} por segundo) com uma resolução $\displaystyle 320px$ x $\displaystyle 240px$.
	
	Com esses mapas de profundidade, a biblioteca \textit{OpenNI} consegue calcular as coodernadas $\displaystle (x,y,z)$ em relação ao \textit{Kinect} de qualquer pixel na imagem. Ou seja, se tivermos a representação de um usuário rastreado na imagem, conseguiremos obter sua localização relativa ao \textit{Kinect} de cada pixel pertencente ao usuário. Então, fixando a posição do \textit{Kinect} no ambiente, conseguiremos estimar a localização de qualquer usuário rastreado em tempo real.
