\section{Módulo de Rastreamento}

	O Módulo de Rastreamento será responsável por rastrear os usuários no
	ambiente, determinar a sua localização física em relação ao \textit{Kinect} e
	gerenciar suas identidades. Para realizar o rastreamento e localização dos
	usuários é utilizado a biblioteca \textit{OpenNI} (\textit{Open Natural
	Interaction}). Trata-se de um \textit{framework} que define \textit{APIs} para
	o desenvolvimento de aplicações de interação natural. Utilizando as imagens de
	profundidade, a detecção e o rastreamento são feito utilizando subtração de
	fundo, descrito na Seção~\ref{sec:deteccao-objeto}, e os objetos detectados são
	representados por suas silhuetas, descrita na
	Seção~\ref{sec:representacao-objeto}.

	As imagens utilizadas para o rastreamento são imagens de profundidade,
	exemplificada na Figura~\ref{fig:depthmaps}, providas pelo \textit{Kinect} que
	são obtidas utilizando o método de Luz Estruturada descrito na
	Seção~\ref{sec:luz-estruturada}. Tais imagens de profundidade nada mais são
	que \textit{depth maps} (mapas de profundidade), em que cada pixel da imagem
	contém o valor estimado da distância em relação ao sensor. O \textit{Kinect}
	fornece esses dados a uma taxa de $\displaystyle 30 fps$ (\textit{frames} por
	segundo) com uma resolução $\displaystyle 640px$ x $\displaystyle 480px$.
	

	\begin{figure}[H]
		\begin{center}
			\includegraphics[scale=0.4]{figuras/4.ProblemaEProposta/mapa-profundidade.png}
		\end{center}
		\caption{Exemplo de uma imagem de profundidade fornecida pelo \textit{Kinect}.}
		\label{fig:depthmaps}
	\end{figure}

	Utilizando os mapas de profundidade é possível calcular as coordenadas$\displaystle (x,y,z)$, em relação ao sensor, de qualquer pixel da imagem. Dessa forma, as coordenadas de uma usuário em relação ao \textit{Kinect} são estimadas utilizando os valores de profundidade referente ao pixel que representa seu centro de massa geométrico. Sendo assim, ao fixar a posição do \textit{Kinect} no ambiente, é possível estimar a localização de qualquer usuário rastreado em tempo real. A Figura~\ref{fig:localizacao} mostra um usuário rastreado pelo Sistema TRUE onde os valores das coordenadas $\displaystle (x,y,z)$ estão em milímetros.

	A Figura~\ref{fig:processo-rastreamento} mostra o fluxo básico do Módulo de Rastreamento.

	\begin{figure}[H]
		\begin{center}
			\includegraphics[scale=0.45]{figuras/4.ProblemaEProposta/localizacao.png}
		\end{center}
		\caption{Imagem do Sistema TRUE de um usuário rastreado e localizado.}
		\label{fig:localizacao}
	\end{figure}
	
	
	\begin{figure}[H]
		\begin{center}
			\includegraphics[scale=0.40]{figuras/4.ProblemaEProposta/diagrama-rastreamento.png}
		\end{center}
		\caption{Representação das etapas propostas para o rastreamento.}
		\label{fig:processo-rastreamento}
	\end{figure}
