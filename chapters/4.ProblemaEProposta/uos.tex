\section{Middleware \textit{UbiquitOS}}
\label{uos}

	O Middleware \textit{UbiquitOS}\footnote{Este trabalho é parte integrante do projeto \textit{UbiquitOS}} é um projeto do grupo de pesquisa \textit{UnBiquitous} da Universidade de Brasília cujo foco está na adaptabilidade de serviços.

	O \textit{UbiquitOS} é uma implementação de um middleware para auxílio de aplicações para ambiente de computação ubíquoa, cuja proposta é compatível com os conceitos apresentados pela \textit{DSOA} (\textit{Device Service Oriented Architecture})~\cite{fabriciobuzzeto}. Ele é visto como um componente de auxílio para desenvolvimento de drivers de recurso, aplicações e \textit{plugins} de rede a serem utilizados em ambientes ubíquos~\cite{fabriciobuzzeto}.

	Um recurso é um grupo de funcionalidades de um dispositivo logicamente relacionadas acessíveis através de interfaces pré-definidas. Tais funcionalidades, por sua vez, são representadas no ambiente através de serviços relacionados~\cite{fabriciobuzzeto}.

	Uma aplicação é a implementação de um conjunto de comportamentos e regras relacionadas ao ambiente inteligente, cujo o objetivo é a tomada de ação ou a interação junto ao usuário. As aplicações ficam no dispositivos do ambiente e se utilizam dos recursos e serviços do mesmo durante a execução.

