\chapter{Sistema \textit{True}}

Observando a realidade de um ambiente inteligente fica claro que as informações como posição das pessoas e suas respectivas identidades são imprescindíveis para que decisões possam ser tomadas. Atualmente, a maioria das soluções encontradas para fornecer esse tipo de informação foram projetadas para funcionar em ambientes rigidamente definidos. Com isso, não seria adequado tentar incorporar soluções como estas em um ambiente com diferentes dimensões, condições de iluminação, posição dos móveis, diferentes sensores pois este resultaria em um cenário diferente. Além disso, a solução deve ser integrada com o middleware \textit{UbiquitOS}, o qual gerência os serviços providos pelo ambiente inteligente.

Esse trabalho propõe, então, um sistema aberto de rastreamento, localização e identificação de pessoas em um ambiente inteligente integrado com o middleware \textit{UbiquitOS}. Tal sistema será chamado de TRUE, \textit{\textbf{T}racking and \textbf{R}ecognizing \textbf{U}sers in the \textbf{E}nvironment}.

O Sistema \textit{True} é um sistema monomodal que utiliza somente dados visuais, como imagens de cor e profundidade. As imagens de profundidade serão utilizadas no rastreamento e localização dos usuários no ambiente, e as imagens de cor serão utilizadas no reconhecimento facial e no cadastro dos usuários. Portanto, os dispositivos presentes no ambiente devem ser capazes de fornecer esses tipos de dados a um taxa e qualidade adequada. 

Para obter os dados necessários, o sistema utiliza o sensor \textit{Kinect} da Microsoft descrito no Apêndice~\ref{sec:kinect}, um dispositivo bastante acessível e capaz de fornecer imagens de cor e de profundidade sincronizadas a uma taxa e qualidade necessária.

O sistema \textit{True} é dividido em quatro módulos principais:


	\begin{itemize}
		\item \textbf{Módulo de Rastreamento e Localização}: parte do sistema responsável pelo rastreamento e localização dos usuários no ambiente.
		\item \textbf{Módulo de Reconhecimento}: parte do sistema responsável por identificar os usuários rastreados.
		\item \textbf{Módulo de Registro}: parte do sistema responsável pelo cadastro de novos usuários e treino do sistema.
		\item \textbf{Módulo \textit{UbiquitOS}}: parte do sistema responsável pela integração e comunicação do sistema com o middleware.
	\end{itemize}

O Módulo de Registro é independente dos demais. Porém, os outros dois módulos devem trocar informações entre si para centralizar todas as informações (localização e reconhecimento) de todos usuários rastreados no ambiente. A seguir, é explicado mais detalhadamente cada módulo e como tal troca de informações ocorre.



	\section{Módulo de Reconhecimento}

	O Módulo de Reconhecimento é responsável pela identificação dos usuários no ambiente utilizando a face como característica biométrica. A face foi escolhida pois ela permite um reconhecimento não intrusivo, como mencionada na Seção~\ref{sec:biometria}. A detecção e o reconhecimento são feitos em imagens de usuários que são passadas pelo Módulo de Rastreamento. Tais imagens são compostas somente pela região em que o usuário se encontra, como mostrado na Figura~\ref{fig:users-img}.

	Basicamente, o processo de reconhecimento é realizado pelas seguintes etapas e ilustrado na Figura~\ref{fig:processo-reconhecimento}:

		\begin{figure}[htb]
			\begin{center}
				\includegraphics[scale=0.3]{figuras/4.ProblemaEProposta/users-img.png}
			\end{center}
			\caption{Exemplo de uma imagem composta somente pela região em que o usuário se encontra.}
			\label{fig:users-img}
		\end{figure}

		\begin{enumerate}
			\item Obtém a imagem de entrada correspondente a imagem formada somente pelo usuário cujo reconhecimento foi requisitado.
			\item Pré-processamento da imagem: a imagem é convertida em escala de cinza.
			\item Realiza detecção facial na imagem. Caso nenhuma face seja encontrada, retorna ``vazio''. Vale ressaltar que no máximo uma face pode ser encontrada nesta imagem.
			\item Processamento da imagem: uma nova imagem é criada recortando a região da face encontrada, a imagem, então, é redimensionada e equalizada criando assim um padrão de tamanho, brilho e contraste nas imagens aumentando a acurácia do reconhecimento.
			\item Reconhecimento facial com \textit{Eigenfaces} é realizado.
			\item Retorna o nome da face ``mais parecida'' e a confiança do reconhecimento.
		\end{enumerate}

		\begin{figure}[htb]
			\begin{center}
				\includegraphics[scale=2.0]{figuras/4.ProblemaEProposta/reconhecimento-simples.png}
			\end{center}
			\caption{Módulo de Reconhecimento do Sistema TRUE.}
			\label{fig:processo-reconhecimento}
		\end{figure}

	% O Módulo de Reconhecimento é dependente do de Rastreamento. Ele ficará ocioso
	% até que chegue uma requisição de reconhecimento de um determinado usuário. A
	% Seção~\ref{sec:rastreamento-reconhecimento} explica mais detalhadamente a
	% relação entre os dois módulos.

	\subsection{Pré-processamento e Processamento da Imagem}
		
		As etapas de processamento das imagens permitem criar um padrão nas mesmas aumentando a acurácia do reconhecimento. No Sistema TRUE as etapas de processamento consistem em converter a imagem em escala de cinza, recorta-la, redimensiona-la e equaliza-la criando, assim, um padrão de cor, tamanho, brilho e contraste nas imagens.

		A Figura~\ref{fig:greyscale} exemplifica uma imagem normal de uma face,
		depois a mesma convertida em escala de cinza e equalizada O
		Apêndice~\ref{apend:processamento} mostra trechos de código em linguagem C
		que implementam tais etapas.

		\begin{figure}[htb]
			\begin{center}
				\includegraphics[scale=0.7]{figuras/4.ProblemaEProposta/greyscale.png}
			\end{center}
			\caption{Exemplo de uma imagem de face normal, em escala de cinza e equalizada. Adaptada de~\cite{shervin}.}
			\label{fig:greyscale}
		\end{figure}

	\subsection{Detecção Facial}

		A detecção facial foi desenvolvida utilizando o método \textit{Viola-Jones}~\ref{ref:viola-jones}. Um método que pode ser utilizado para construir uma abordagem de detecção facial rápida e eficaz~\cite{violajones} em tempo real. Além disso, este método é implementado pela biblioteca \textit{OpenCV} (\textit{Open Source Computer Vision}) onde bons classificadores em cascata de características \textit{Haar} são fornecidos.

		Basicamente, o processo de detecção facial procura por uma face em uma imagem pré-processada. Para realizar detecção facial utilizando o método \textit{Viola-Jones} é necessário a utilização de um classificador em cascata, como mencionado na Subseção~\ref{subsec:reconhecimento}. Portanto, entre os diversos classificadores em cascata presentes na biblioteca \textit{OpenCV}, foi utilizado o classificador \textit{haarcascade\underline{ }frontalface\underline{ }alt.xml}, um classificador treinado para detectar faces frontais em imagens.

		% A Figura~\ref{fig:diagrama-deteccao} mostra o fluxo básico do processo de detecção de faces no Sistema TRUE.
		% 	\begin{figure}[H]
		% 	\begin{center}
		% 		\includegraphics[scale=0.5]{figuras/4.ProblemaEProposta/diagrama-detectar-face.png}
		% 	\end{center}
		% 	\caption{Fluxo de execução do processo de detecção facial no Sistema TRUE.}
		% 	\label{fig:diagrama-deteccao}
		% \end{figure}

		O processo de básico de detecção de faces no Sistema TRUE possui as seguintes etapas:

		\begin{enumerate}
			\item Lê um classificador treinado para detectar faces em uma image.
			\item Obtém a imagem de entrada. Tal imagem é composta somente pelo usuário, como mostrado na Figura~\ref{fig:users-img}, cujo reconhecimento foi requisitado, além de estar em escala de cinza.
			\item Utilizando o classificador, tentar obter uma face na imagem.
			\item Retorna a região da face detectada ou retorna ``vazio'' caso nenhuma face tenha sido encontrada. Vale salientar que existe no máximo uma face na imagem, pois contém somente um usuário.
		\end{enumerate}

	\subsection{Reconhecimento Facial com \textit{Eigenfaces}}

		O reconhecimento facial foi desenvolvido utilizando
		\textit{Eigenfaces}~\ref{sec:reconhecimento}, e teve como base algumas etapas descritas em~\cite{shervin}. Consiste em uma técnica bastante satisfatória quando utilizada sobre uma base de dados (faces) relativamente grande, permitindo ao sistema inferir, das imagens suas principais características e, partindo delas, realizar o reconhecimento das imagens utilizando um número bastante reduzido de cálculos~\cite{artigo-eigenface}, permitindo, assim, um reconhecimento em tempo real.

		A base de dados utilizada no Sistema TRUE é formada por imagens no formato PGM (\textit{Portable Gray Map}) com tamanho de 92x112 pixels e em escala de cinza. A base é composta por um banco de faces de alunos de Ciência da Computação da Universidade de Brasília e por um banco de imagens de faces da Universidade de Cambridge~\cite{cambridgeFaceDb}, mostrado na Figura~\ref{fig:cambridgeFaceDb}. Este último, é formado por imagens de faces de 40 pessoas diferentes. Para cada pessoa, existem 10 diferentes imagens tiradas em diferentes épocas, com diferentes condições de iluminação, com diferentes expressões faciais (olhos abertos e fechados, sorrindo e não sorrindo, entre outros) e com diferentes detalhes faciais (óculos, sem óculos). 

		\begin{figure}[htb]
			\begin{center}
				\includegraphics[scale=0.4]{figuras/4.ProblemaEProposta/cambrigdefacedb.png}
			\end{center}
			\caption{Banco de imagens de faces da Universidade de Cambridge~\cite{cambridgeFaceDb}.}
			\label{fig:cambridgeFaceDb}
		\end{figure}

		A Figura~\ref{fig:diagrama-reconhecimento} mostra o fluxo básico do processo de reconhecimento facial no Sistema TRUE.

			\begin{figure}[htb]
			\begin{center}
				\includegraphics[scale=0.5]{figuras/4.ProblemaEProposta/diagrama-reconhecimento2.png}
			\end{center}
			\caption{Fluxo de execução do processo de reconhecimento facial no Sistema TRUE.}
			\label{fig:diagrama-reconhecimento}
		\end{figure}

		% A primeira etapa consiste na leitura dos dados de treinamento. Esses dados são compostos pela lista dos nomes e imagens das faces dos usuários cadastrados no sistema, pelo vetor de \textit{Eigenfaces}, pela \textit{Eigenface} média e pelos \textit{eigenvalues}.

		% Uma das etapas intermediárias consiste no cálculo da distância entre a imagem projetada no subespaço PCA aos eigenfaces. Inicialmente, o cálculo desta distância era feito utilizando distância euclidiana, porém não apresentava bons resultados em algumas condições. Portanto, este cálculo passou a ser feito utilizando distância Mahalanobis. Contudo, ela também não apresentou bons resultados em alguns casos. Então, alguns testes foram feitos utilizando as duas distâncias de maneira conjunta: uma imagem só é tida como reconhecida quando o resultado das duas distâncias apontarem para a mesma identidade. Com isso, houve uma melhora significativa dos resultados.

		Uma das etapas intermediárias consiste no cálculo da distância entre a imagem projetada no subespaço PCA aos \textit{eigenfaces}. Inicialmente, o cálculo desta distância era feito utilizando distância Euclidiana. Contudo, testes foram realizados com alguns usuários, em que o sistema realizava 20 tentativas de reconhecimento de usuários, e os resultados não foram satisfatórios. Portanto, os mesmos testes foram realizados utilizando distância Mahalanobis. Porém, os resultados também não foram satisfatórios. Então, os mesmos testes foram feitos utilizando as duas distâncias de maneira conjunta: uma imagem só é tida como reconhecida quando o resultado das duas distâncias apontarem para a mesma identidade. Com isso, houve uma melhora significativa dos resultados. Os resultados destes testes são mostrados na Tabela~\ref{tab:distancias}, onde fica claro a melhora dos resultados quando se utiliza ambas distâncias. Nessa tabela, os nome Pessoa1, Pessoa2, Pessoa3, são nomes dados as pessoas cujas fotos estão presentes no banco de imagens de faces da Universidade de Cambrige~\cite{cambridgeFaceDb}.

		\begin{table}[H]
		\begin{center}
			\caption{Resultados do teste de reconhecimento feito com o usuário Danilo utilizando as diferentes distâncias.}
			\label{tab:distancias}
			\begin{tabular}{|c|c|c|c|c|c|c|c|}
				\hline & \bf Danilo & \bf Pedro & \bf Ana & \bf Pessoa1 & \bf Pessoa2 & \bf Pessoa3 & \bf Desconhecido\\
				\hline \bf Euclidiana & 45\% & 40\% & & 5\% & 5\% & 5\% &\\
				\hline \bf Mahalanobis & 40\% & & 15\% & 35\% & 10\% & &\\
				\hline \bf Ambas Distâncias & 75\% & & & & & & 25\%\\
				\hline
			\end{tabular}
		\end{center}
	\end{table}

		A última etapa consiste no cálculo da confiança do reconhecimento. Este cálculo foi feito utilizando a distância da imagem de entrada do usuário à imagem mais similar das imagens de treinamento. O valor da confiança varia de 0.0 a 1.0, em que uma confiança de 1.0 significaria uma ``correspondência perfeita''. A fórmula utilizada para calcular a confiança é uma métrica muito básica que não necessessariamente é muito real, dada pela Fórmula~\ref{eq:confianca}~\cite{shervin}. Nesta fórmula $\displaystyle d_e$ é a distância da imagem de entrada do usuário a imagem mais similar das imagens de treinamento, $\displaystyle n_t$ é o número de imagens utilizadas no treinamento e $\displaystyle n_e$ é o número de \textit{eigenfaces}.


		\begin{equation}
			\label{eq:confianca}
			Confianca = 1 - \frac{\sqrt{\frac{d_e}{n_t * n_e}}}{255}
		\end{equation}






















	\section{Módulo de Rastreamento}

	O Módulo de Rastreamento será responsável por rastrear os usuários no ambiente inteligente, determinar a sua localização física em relação ao \textit{Kinect} e gerenciar suas identidades. Para realizar o rastreamento e localização dos usuários é utilizado a biblioteca \textit{OpenNI} (\textit{Open Natural Interaction}). Trata-se de um \textit{framework} que define \textit{APIs} para o desenvolvimento de aplicações de interação natural. Utilizando as imagens de profundidade, a detecção e o rastreamento são feito utilizando subtração de fundo, descrito na Seção~\ref{sec:deteccao-objeto}, e os objetos detectados são representados por suas silhuetas, descrita na Seção~\ref{sec:representacao-objeto}.

	As imagens utilizadas para o rastreamento são imagens de profundidade, exemplificada na Figura~\ref{fig:depthmaps}, providas pelo \textit{Kinect} que são obtidas utilizando o método de Luz Estruturada descrito na Seção~\ref{sec:luz-estruturada}. Tais imagens de profundidade nada mais são que \textit{depth maps} (mapas de profundidade), em que cada pixel da imagem contém o valor estimado da distância em relação ao sensor. O \textit{Kinect} fornece esses dados a uma taxa de $\displaystyle 30 fps$ (\textit{frames} por segundo) com uma resolução $\displaystyle 640px$ x $\displaystyle 480px$.
	

	\begin{figure}[H]
		\begin{center}
			\includegraphics[scale=0.45]{figuras/4.ProblemaEProposta/mapa-profundidade.png}
		\end{center}
		\caption{Exemplo de uma imagem de profundidade fornecida pelo \textit{Kinect}.}
		\label{fig:depthmaps}
	\end{figure}

	Utilizando os mapas de profundidade é possivel calcular as coodernadas $\displaystle (x,y,z)$, em relação ao sensor, de qualquer pixel da imagem. Dessa forma, a posição de um usuário rastreado é determinada utilizando as coordenadas presentes no pixel que representa seu centro geométrico. Sendo assim ao fixar a posição do \textit{Kinect} no ambiente, conseguiremos estimar a localização de qualquer usuário rastreado em tempo real. A Figura~\ref{fig:localizacao} mostra um usuário rastreado pelo Sistema TRUE onde suas coordenadas em relação ao \textit{Kinect} foram estimadas utilizando os valores de profundidade referente ao pixel que representa seu centro de massa geométrico. Os valores das coordenadas $\displaystle (x,y,z)$ estão milimetros.

	\begin{figure}[H]
		\begin{center}
			\includegraphics[scale=0.45]{figuras/4.ProblemaEProposta/localizacao.png}
		\end{center}
		\caption{Imagem do Sistema TRUE de um usuário rastreado e localizado.}
		\label{fig:localizacao}
	\end{figure}
	
	
	\begin{figure}[H]
		\begin{center}
			\includegraphics[scale=0.45]{figuras/4.ProblemaEProposta/Rastreamento.png}
		\end{center}
		\caption{Representação das etapas propostas para o rastreamento.}
		\label{fig:processo-rastreamento}
	\end{figure}

	\section{Relação Rastreamento e Reconhecimento}
\label{sec:rastreamento-reconhecimento}

	Até então, foi descrito como os Módulos de Rastreamento e de Reconhecimento funcionam de maneira isolada. Entretando, eles operam em conjunto. O Módulo de Rastreamento detém as informações sobre todos os usuários rastreados no ambiente e é responsável por requisitar identificação ao Módulo de Reconhecimento. Isto acontece sempre que um novo usuário for detectado ou quando for necessário reconhecer novamente um usuário já rastreado.

	% Basicamente, quando um novo usuário for detectado, a relação entre rastreamento e reconhecimento acontecerá de acordo com as etapas descritas na Figura~\ref{fig:rastreamento-reconhecimento}.

	Quando um novo usuário for detectado, o Módulo de Rastreamento obtém 20 imagens de cor sucessivas deste usuário. Para cada imagem, ele cria uma nova imagem composta somente pela região em que este usuário se encontra, como mostrado na Figura~\ref{fig:users-img}, e a envia ao Módulo de Reconhecimento requisitando identificação ao mesmo. O Módulo de Reconhecimento, por sua vez, realiza o reconhecimento facial e retorna ao Módulo de Rastreamento o nome do usuário e a confiança obtida. A cada 0.5 segundos, o Módulo de Reconhecimento verifica se chegou algum resultado de identificação. Caso tenha chegado, o Módulo de Rastreamento computa o resultado e decide qual identidade será atribuída ao respectivo usuário. A Figura~\ref{fig:rastreamento-reconhecimento} ilustra esta relação entre os dois módulos.

		\begin{figure}[htb]
			\begin{center}
				\includegraphics[scale=0.5]{figuras/4.ProblemaEProposta/diagrama-relacao.png}
			\end{center}
			\caption{Representação da relação que o Módulo de Rastreamento terá com o Módulo de Reconhecimento quando um novo usuário for detectado.}
			\label{fig:rastreamento-reconhecimento}
		\end{figure}
	
		% \begin{enumerate}
		%  	\item O Módulo de Rastreamento detecta novo usuário, e obtém um número pré-definido de imagens sucessivas do novo usuário. Para cada imagem, ele cria uma nova imagem de cor contendo somente aquele usuário, como mostrado na Figura (\textbf{colocar a figura aqui}), e a envia para o Módulo de Reconhecimento.
		%  	\item Para cada imagem recebida, o Módulo de Reconhecimento tenta reconhecer o novo usuário e retorna ``vazio'' ou o nome e a confiança do reconhecimento.
		%  	\item O Módulo de Rastreamento verifica se a confiança é maior que um limiar pré-definido, se for ele incrementa o contador que armazena o número de vezes que o usuário foi reconhecido, armazena o nome obtido juntamente com a confiança e calcula qual nome  será atribuído ao novo usuário. Esse cálculo será feito por meio de uma média ponderada utilizando os diferentes resultados obtidos por cada reconhecimento e suas respectivas confianças.
	 % 	\end{enumerate} 
	
	Ao invés de realizar o reconhecimento somente quando novos usuários são detectados, com o objetivo de melhorar a confiança no reconhecimento, o Sistema TRUE realiza continuamente a identificação dos usuários já reconhecidos. Essas tentativas de reconhecer novamente os usuários ocorrerão a cada 5 segundos seguindo as mesmas etapas de quando um novo usuário for detectado. A única etapa que se difere é a primeira, ou seja, ao invés de obter várias imagens de um mesmo usuário, é obtida uma imagem de cada usuário rastreado e a mesma é enviada ao Módulo de Reconhecimento.

	Como visto na Figura~\ref{fig:rastreamento-reconhecimento}, ao obter um resultado de reconhecimento para determinado usuário, o Módulo de Rastreamendo deve computar qual identidade será atribuída ao mesmo. Para isso, este módulo mantém para cada usuário o número total de vezes que este já foi reconhecido, os diferentes nomes obtidos pelo Módulo de Reconhecimento bem como a confiança média para cada nome e o número de vezes que cada nome foi atribuído àquele usuário. Com todos esses dados, a identidade do usuário é definida por meio de uma média ponderada entre as confianças de cada nome, onde os pesos utilizados são compostos pelo número de vezes que o respectivo nome foi atribuído ao usuário. A Equação~\ref{eq:media_ponderada} mostra como é calculada essa média ponderada, em que $\displaystyle N_i$ é o número de vezes que o nome foi atribuído ao usuário e $\displaystyle C_i$ é a confiança média. O exemplo a seguir aprensenta em detalhes como a identidade de um usuário é definida.

	\begin{equation}
		\label{eq:media_ponderada}
		M_p = \frac{N_1 * C_1 + N_2 * C_2 + ... + N_n * C_n}{N_1 + N_2 + ... + N_n}
	\end{equation}


	\begin{description}
 		João é um usuário do ambiente inteligente que está sendo rastreado há algum tempo e que já foi reconhecido algumas vezes. Como já mencionado, o Módulo de Rastreamento mantém para cada usuário o número total de vezes que este já foi reconhecido, os diferentes nomes obtidos pelo Módulo de Reconhecimento bem como a confiança média para cada nome e o número de vezes que cada nome foi atribuído àquele usuário. Neste caso, a Tabela~\ref{tab:joao} mostra os dados mantidos para o usuário João pelo Módulo de Rastreamento, em que ``João'', ``Danilo'' e ``Tales'' são as identidades que já foram obtidas ao se tentar reconhecer o João. A segunda coluna desta tabela mostra a confiança média do reconhecimento para cada nome e a última mostra o número de vezes que cada nome foi atribuído a João. 

 		Para decidir qual identidade será atribuída ao usuário João primeiramente calcula-se a média ponderada como mostrado nos cálculos~\ref{eq:joao}. Então, a identidade atribuída será aquela cuja confiança média mais se aproxima da média ponderada. Neste caso, a identidade atribuída a João será ``João'', como mostrado nos cálculos~\ref{eq:joao2}.

 		% No momento, o Módulo de Rastreamento mantém vários dados sobre o João descritos na Tabela~\ref{tab:joao}. Como já mencionado, o Módulo de Rastreamento Então, para computar qual identidade será atribuída ao usuário, o Módulo de Rastreamento realiza os cálculos~\ref{eq:joao}. Como o resultado da média ponderada se aproxima mais da confiança do nome João, este foi escolhido como sendo sua identidade.
	\end{description}

	\begin{table}[htb]
		\begin{center}
			\caption{Exemplos de dados de reconhecimento mantidos para cada usuário rastreado pelo Módulo de Rastreamento.}
			\label{tab:joao}
			\begin{tabular}{|c|c|c|}
				\hline \bf Nome & Confiança Média & Número de Vezes \\
				\hline \hline \bf João & 0.947302 & 15 \\
				\hline \bf  Danilo & 0.934010 & 1 \\
				\hline \bf Tales & 0.950320 & 3 \\
				\hline
				\hline \multicolumn{2}{|c|}{\bf Total}  & 19 \\
				\hline
			\end{tabular}
		\end{center}
	\end{table}

	\begin{align}
		\label{eq:joao}
		M_p = \frac{15 * 0.947302 + 1 * 0.934010 + 3 * 0.950320}{19} = 0.947079
	\end{align}

	\begin{align}
		\label{eq:joao2}
	 	& Joao: 0.947079 - 0.947302 = -0.000223\\
		\nonumber & Danilo: 0.934010 - 0.947302 = -0.013292\\
		\nonumber & Tales: 0.950320 - 0.947302 = 0.003018
	\end{align}

	Com todos esses dados de reconhecimento, além dos dados de rastreamento e localização de cada usuário, o Módulo de Rastreamento centraliza todas as informações necessárias de cada usuário. A Figura~\ref{fig:truetotal} exemplifica um usuário rastreado pelo Sistema TRUE onde as informações de localização e identificaçao estão presentes.

	\begin{figure}[htb]
			\begin{center}
				\includegraphics[scale=0.5]{figuras/4.ProblemaEProposta/user-reconhecido.png}
			\end{center}
			\caption{Exemplo de um usuário rastreado e identificado pelo Sistema TRUE.}
			\label{fig:truetotal}
		\end{figure}


	\section{Módulo de Registro}

	O Módulo de Registro é responsável por cadastrar novos usuários no sistema e treiná-lo para também reconhecer esse novo usuário. Basicamente, o processo de registro, ilustrado na Figura~\ref{fig:registro}, segue as seguintes etapas :

		\begin{figure}[H]
			\begin{center}
				\includegraphics[scale=1.5]{figuras/4.ProblemaEProposta/registro.png}
			\end{center}
			\caption{Módulo de Registro do Sistema TRUE.}
			\label{fig:registro}
		\end{figure}		

		\begin{enumerate}
			\item O novo usuário fica em uma posição fixa e frontal em relação ao \textit{Kinect}. 
			\item O sistema obtém seis imagens frontais do usuário.
			\item O usuário, então, deve rotacionar um pouco a face para a esquerda e o sistema obtém mais duas imagens do usuário. Depois, deve rotacionar um pouco para direita e o sistema obtém outras duas imagens do usuário.
			\item As imagens obtidas são processadas: as imagens são convertidas em escala de cinza, novas imagens são criadas recortando a região da face encontrada, as imagens, então, são redimensionadas e equalizadas criando assim uma padrão de tamanho, brilho e contraste nas imagens.
			\item Armazena-se as imagens.
			\item O sistema é treinado para, também, reconhecer esse usuário.
		\end{enumerate}

	Após o treinamento, o Sistema TRUE reiniciará para que o reconhecimento seja feito utilizando as novas informações obtidas com o treinamento.

	Como descrito, durante o processo de captura das imagens, o usuário deve rotacionar a face um pouco para direita e para esquerda obtendo, além de imagens frontais, imagens um pouco mais de perfil do usuário, como mostrado na Figura~\ref{fig:imgs-cadastro}. Isso foi feito para que o sistema se torne um pouco mais robusto em relação a posição da face dos usuários ao tentar reconhece-los.

		\begin{figure}[H]
			\begin{center}
				\includegraphics[scale=0.4]{figuras/4.ProblemaEProposta/face-registro.png}
			\end{center}
			\caption{Exemplo de imagens resultantes do cadastro do usuário.}
			\label{fig:imgs-cadastro}
		\end{figure}	

	A última etapa do módulo de registro consiste no treinamento do sistema, descrito na Figura~\ref{fig:treinamento}.

		\begin{figure}[hbt]
			\begin{center}
				\includegraphics[scale=0.7]{figuras/4.ProblemaEProposta/diagrama-registro.png}
			\end{center}
			\caption{Fluxo básico do treinamento do Sistema TRUE.}
			\label{fig:treinamento}
		\end{figure}	



	\section{\textit{SmartSpace} Laico}

		O ambiente para o qual o Sistema TRUE será projetado, desenvolvido e testado
		chama-se LAICO (\textbf{LA}boratório de sistemas \textbf{I}ntegrados e
		\textbf{CO}ncorrente), um laboratório do Departamento de Ciência da Computação
		da Universidade de Brasília. O LAICO possui dimensões de, aproximadamente, 
		$\displaystyle 7,67m$ x $\displaystyle 6,45m$ ilustrado pela
		Figura~\ref{fig:laico}.
	
		\begin{figure}[hbt]
			\begin{center}
				\includegraphics[scale=0.6]{figuras/4.ProblemaEProposta/laico.png}
			\end{center}
			\caption{Planta do \textit{SmartSpace} Laico.}
			\label{fig:laico}
		\end{figure}	


\section{Módulo \textit{UbiquitOS}}














