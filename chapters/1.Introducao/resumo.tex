\begin{resumo}

	Mark Weiser introduziu o conceito de Computação Ubíqua em 1991 que consiste em uma nova linha de pesquisa tecnológica caracterizada por um ambiente que interage de maneira inteligente com seus usuários e possui uma ampla e transparente interação entre dispositivos e serviços disponíveis. Para que os ambientes inteligentes, \textit{smart spaces}, agreguem as tarefas do usuários, a inteligência presente deve coordená-los de acordo com as informações sobre o ambiente e o usuário. Portanto, informações como posição do usuário e sua respectiva identidade são de grande valia para tornar possível a interação entre o usuário e os recursos presentes.

	Esse trabalho corresponde a uma primeira iniciativa de disponibilizar informações como estas nos ambientes inteligentes gerenciados pelo middleware \textit{uOS}. Para isso, é proposto um sistema de reconhecimento facial, rastreamento e localização de usuários em um ambiente inteligente. Tal sistema é chamado de TRUE (\textit{Tracking and Recognizing Users in the Environment}) que utiliza imagens de cor e de profundidade como dados de entrada.

\end{resumo}

\selectlanguage{american}

\begin{abstract}
	
	Mark Weiser introduced the concept of Ubiquitous Computing in 1991 consisting of a new line of technological research characterized by an intelligent environment that acts with users and has an extensive and transparent interaction between available devices and services. Looking at the reality of such environment, it is clear that informations such as user's positions and identities are very valueable to make possible the interaction between users and resources in the environment.

	Therefore, this work proposes a face recognition, tracking and localization system in order to provide context information containing the identity and postion of each user in an intelligent environment. This system is called TRUE (Tracking and Recognizing Users in the Environment) and uses color and depth images as input data.

\end{abstract}

\selectlanguage{brazil}
