\begin{resumo}

	Mark Weiser introduziu o conceito de Computação Ubíqua em 1991 que consiste em uma nova linha de pesquisa tecnológica caracterizada por um ambiente que interage de maneira inteligente com seus usuários e possui uma ampla e transparente interação entre dispositivos e serviços disponíveis. 

% 

	Observando a realidade de um ambiente como esse, fica claro que informações como posição dos usuários e suas respectivas identidades são de grande valia para tornar possível a interação entre os usuários e os recursos presentes no ambiente.



	Esse trabalho propõe, então, um sistema de reconhecimento facial, rastreamento e localização de usuários em um ambiente inteligente afim de prover informações de contexto contendo as identidades e as localizações de cada usuário no ambiente. Tal sistema é chamado de TRUE (\textit{Tracking and Recognizing Users in the Environment}) que utiliza imagens de cor e de profundidade como dados de entrada.

\end{resumo}

\selectlanguage{american}

\begin{abstract}
	
	Mark Weiser introduced the concept of Ubiquitous Computing in 1991 consisting of a new line of technological research characterized by an intelligent environment that acts with users and has an extensive and transparent interaction between available devices and services. Looking at the reality of such environment, it is clear that informations such as user's positions and identities are very valueable to make possible the interaction between users and resources in the environment.

	Therefore, this work proposes a face recognition, tracking and localization system in order to provide context information containing the identity and postion of each user in an intelligent environment. This system is called TRUE (Tracking and Recognizing Users in the Environment) and uses color and depth images as input data.

\end{abstract}

\selectlanguage{brazil}
