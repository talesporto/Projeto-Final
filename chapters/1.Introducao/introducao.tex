
\chapter{Introdução}
	
A Computação Ubíqua, \textit{ubicomp}, a tempos vem sendo tema de diversas pesquisas ao redor do mundo. Mark Weiser diz que o computador do futuro deve ser algo invisível proporcionando ao usuário um melhor foco na tarefa e não na ferramenta. A Computação Ubíqua tenta atribuir tal invisibilidade aos computadores criando uma realidade na qual o computador se integra a ambientes físicos constituindo os ambientes inteligentes, \textit{smart-spaces}, que interagem com seus usuários e possuem uma ampla e transparente interação entre dispositivos e serviços disponíveis~\cite{fabriciobuzzeto,alegomes,weiser1, weiser2}.

Observando a realidade de um ambiente como esse, fica claro que informações como posição dos usuários e suas respectivas identidades são de grande valia para tornar possível a interação entre os usuários e os recursos presentes. Além disso, tais informações garantem uma maior acurácia nas tomadas de decisões. Um ambiente ubíquo capaz de obter tais informações, pode prover uma personalização automática de acordo com as preferências do usuário e até mesmo prover um ambiente mais seguro com controle de acesso físico e prevenção de fraudes \cite{saocarlos}. Contudo, informações como estas são muito complicadas de se obter devido a alta dinamicidade do ambiente, no qual usuários entram e saem a todo momento e interagem entre si e com diversos equipamentos.




% A identificação de um usuário em um \textit{SmartSpace} é feita por meio de sistema de reconhecimento automático. Há alguns anos, um grande número de pesquisas vem sendo desenvolvidas para criação deste tipo de sistema  \cite{saocarlos}. Um dos motivos clássicos é que os métodos baseados em cartões de identificação e senhas não são altamente confiáveis. Estes podem ser perdidos, extraviados e até fraudados \cite{bolle}.

% Um ambiente ubíquo capaz de reconhecer seus usuários, pode prover uma personalização automática do ambiente de acordo com as preferências do usuário e até mesmo prover um ambiente mais seguro com controle de acesso físico e prevenção de fraudes \cite{saocarlos}. Atualmente, os métodos de reconhecimento mais utilizados são baseados no uso de cartões magnéticos e senhas, que requerem sua utilização durante uma transação, mas que não verificam sua idoneidade \cite{daugman}.

% Hoje em dia, várias técnicas de reconhecimento por meio de faces, íris, voz, entre outras, vêm sendo estudadas e utilizadas em sistemas de reconhecimento automático \cite{bolle}. O reconhecimento facial pode ser considerada como uma das principais funções do ser humano pois permite identificar uma grande quantidade de faces e aspectos psicológicos demonstrados pela fisionomia. Pode ser considerada, também, como um problema clássico da computação visual pela complexidade existente na detecção e reconhecimento de características e padrões \cite{saocarlos}.

Este trabalho se encontra organizado da seguinte maneira. No Capítulo~\ref{cap:fundamentacao} é apresentada uma abordagem conceitual sobre rastreamento de entidades, localização e identificação das mesmas em um ambiente inteligente, mostrando algumas técnicas e métodos além das principais dificuldades encontradas. No Capítulo~\ref{cap:trabalhos_correlatos} são analisados alguns projetos que focam rastreamento, identificação e localização de pessoas em um ambiente inteligente. O trabalho desenvolvido é apresentado no Capítulo~\ref{cap:true}, em que são descritos as diferentes etapas da implementação do Sistema TRUE e as soluções utilizadas para realizar rastreamento, localização e identificação de pessoas. Neste capítulo também é descrita em detalhes como a integração do sistema com o Middleware \textit{uOS} foi implementada. No Capítulo~\ref{cap:testes} são apresentados os resultados dos testes realizados na implementação do Sistema TRUE. Por fim, no Capítulo~\ref{cap:conclusao} são apresetandos as considerações finais sobre este trabalho bem como os trabalhos futuros.


