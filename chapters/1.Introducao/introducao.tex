
\chapter{Introdução}
	
A computação ubíqua a tempos vem sendo tema de diversas pesquisas ao redor do mundo. Mark Weiser diz que o computador do futuro deve ser algo invisível \cite{weiser1, weiser2} proporcionando ao usuário um melhor foco na tarefa e não na ferramenta. A computação ubíqua tenta atribuir essa invisibilidade aos computadores buscando cada vez mais a diminuição do tamanho, a especificidade da tarefa e se acoplando aos objetos do dia-a-dia.

Um ambiente onde a computação ubíqua acontece em sua totalidade é chamado de SmartSpace \cite{gregoryabowd}. Esse ambiente provê ao usuário uma melhor forma de interagir com os computadores usando diversas tecnologias que estimulam a interatividade natural. Tais tecnologias são capazes de fornecer inteligência, ao SmartSpace, necessária para concretizar a visão da ubicomp \cite{fabriciobuzzeto}.

Para conseguir uma boa interação entre as diversas peças que compõem o SmartSpace é necessário que se tenha a disposição informações de contexto,  como quem está no ambiente, onde está, o que está fazendo e outras que ajudam o sistema a definir o melhor ajuste dos equipamentos. Com uma base rica de informações de contexto, contendo os perfis dos usuários, garantimos uma maior acurácia na tomada de decisões. Informações de contexto como essas são complicadas de se obter devido a alta dinamicidade do ambiente, no qual usuários entram e saem a todo momento e interagem com diversos equipamentos.

A identificação de usuário em um SmartSpace é feita por meio de sistema de reconhecimento automático. Há alguns anos, um grande número de pesquisas vem sendo desenvolvidas para criação sistemas deste tipo \cite{saocarlos}. Um dos motivos clássicos é que os métodos baseados em cartões de identificação e senhas não são altamente confiáveis. Estes podem ser perdidos, extraviados e até fraudados \cite{bolle}.

Um ambiente ubíquo capaz de reconhecer seus usuários, pode prover uma personalização automática do ambiente de acordo com as prefrências de cada usuário e até mesmo prover um ambiente mais seguro com controle de acesso físico e prevenção de fraudes \cite{saocarlos}. Atualmente, os métodos de reconhecimento mais utilizados são baseados no uso de cartões magnéticos e senhas, que requerem sua utilização durante uma transação, mas que não verificam sua idoneidade \cite{daugman}.

Hoje em dia, várias técnicas de reconhecimento por meio de faces, íris, voz, entre outras, vêm sendo estudadas e utilizadas em sistemas de reconhecimento automático \cite{bolle}. O reconhecimento facial pode ser considerada como uma das principais funções do ser humano pois permite identificar uma grande quantidade de faces e aspectos psicológicos demonstrados pela fisionomia. Pode ser considerada, também, como um problema clássico da visão artificial pela complexidade existente na detecção e reconhecimento de características e padrões \cite{saocarlos}.

O reconhecimento facial vem se desenvolvendo junto a ``quarta geração'' de computadores através de sua aplicação na nova geração de interfaces que consiste na detecção e reconhecimento de pessoas \cite{saocarlos}.

É proposta então uma solução para o problema de localização e identificação de perfis de usuários em um SmartSpace utilizando como base o middleware UbiquitOS \cite{alegomes} integrado com o Kinect.

\section{Organização do trabalho}

	Explicar a estrutura da monografia.